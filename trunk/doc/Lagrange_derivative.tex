%% LyX 2.0.3 created this file.  For more info, see http://www.lyx.org/.
%% Do not edit unless you really know what you are doing.
\documentclass[twoside,english]{paper}
\usepackage{lmodern}
\renewcommand{\ttdefault}{lmodern}
\usepackage[T1]{fontenc}
\usepackage[latin9]{inputenc}
\usepackage[a4paper]{geometry}
\geometry{verbose,tmargin=3cm,bmargin=2.5cm,lmargin=2cm,rmargin=2cm}
\usepackage{color}
\usepackage{babel}
\usepackage{float}
\usepackage{bm}
\usepackage{amsthm}
\usepackage{amsmath}
\usepackage{amssymb}
\usepackage{graphicx}
\usepackage{esint}
\usepackage[unicode=true,pdfusetitle,
 bookmarks=true,bookmarksnumbered=false,bookmarksopen=false,
 breaklinks=false,pdfborder={0 0 0},backref=false,colorlinks=false]
 {hyperref}
\usepackage{breakurl}

\makeatletter

%%%%%%%%%%%%%%%%%%%%%%%%%%%%%% LyX specific LaTeX commands.
%% Because html converters don't know tabularnewline
\providecommand{\tabularnewline}{\\}

%%%%%%%%%%%%%%%%%%%%%%%%%%%%%% Textclass specific LaTeX commands.
\numberwithin{equation}{section}
\numberwithin{figure}{section}

%%%%%%%%%%%%%%%%%%%%%%%%%%%%%% User specified LaTeX commands.
\usepackage{babel}

\@ifundefined{showcaptionsetup}{}{%
 \PassOptionsToPackage{caption=false}{subfig}}
\usepackage{subfig}
\makeatother

\begin{document}

\title{Tensor Gluons in {APFEL}}

\author{Valerio Bertone$^{a}$}

\institution{$^{a}$PH Department, TH Unit, CERN, CH-1211 Geneva 23, Switzerland}
\maketitle

\begin{abstract}
In this document I will present the strategy to implement the DGLAP
evolution in {APFEL} in the presence of tensor gluons.
\end{abstract}
\tableofcontents{}

\section{Tensor Gluon Splitting Functions}

When considering the PDF evolution in the presence of tensor gluons we
need consider a generalized prescription to subtract the soft
difergence in in $x=1$ from the plitting functions. In particular, one
of the new typologies of integral we need to consider is:
\begin{equation}
J_{n,m}[f(z)]=\int_x^1 dz\frac{z^nf(z)}{(1-z)_+^m}\,,\qquad n\geq 0\,.
\end{equation}
But using the binomial expansion we can write:
\begin{equation}
z^n = \sum_{i=0}^n (-1)^i {n \choose i} (1-z)^i\,.
\end{equation}
Therefore:
\begin{equation}\label{Jnmfz}
\begin{array}{c}
\displaystyle J_{n,m}[f(z)]=\sum_{i=0}^n (-1)^i {n \choose i} \int_x^1
dz\frac{f(z)}{(1-z)_+^{m-i}}\\
\\
\displaystyle = \sum_{i=0}^{m-1} (-1)^i {n \choose i} \int_x^1
dz\frac{f(z)}{(1-z)_+^{m-i}} + \sum_{j=0}^{n-m}(-1)^{j+m}{n \choose j+m}\int_x^1dzf(z)(1-z)^j\,.
\end{array}
\end{equation}
So, it's clear that, if $n\geq m$, the second sum in the equation
above does not need any +-prescription because it is always
convergent, while the first term is relatively easy to treat, provided
that we know how to compute the derivatives of $f(z)$. We will deat
with this kind of integrals later. Now, to complete the picture, the
second new typoligy of integrals to consider in the presence of tensor
gluons is:
\begin{equation}
J_{-1,m}[f(z)]=\int_x^1 dz\frac{f(z)}{z(1-z)_+^m}\,.
\end{equation}
Now, using the geometrical series, we can write that:
\begin{equation}
\frac1z = \sum_{i=0}^\infty (1-z)^i\,.
\end{equation}
Therefore:
\begin{equation}\label{Jm1mfz}
\begin{array}{c}
\displaystyle J_{-1,m}[f(z)]=\sum_{i=0}^\infty \int_x^1
dz\frac{f(z)}{(1-z)_+^{m-i}}
\displaystyle = \sum_{i=0}^{m-1} \int_x^1
dz\frac{f(z)}{(1-z)_+^{m-i}} +
\sum_{j=0}^{\infty}\int_x^1dzf(z)(1-z)^j\\
\\
\displaystyle = \sum_{i=0}^{m-1} \int_x^1
dz\frac{f(z)}{(1-z)_+^{m-i}} +
\int_x^1dz\frac{f(z)}{z}\,.
\end{array}
\end{equation}
where again the second integral in the equation above is convergent
without the need of any prescriton.

Now, looking at eqs. (\ref{Jnmfz}) and (\ref{Jm1mfz}), we notice that
both contain the same typology of integral which need to be
regularized according to the prescription given in
Ref. \cite{Savvidy:2013gsa}. In particular we need to consider the
following integral:
\begin{equation}
\begin{array}{rl}
L_j[f(z)] &=\displaystyle \int_x^1 dz\frac{f(z)}{(1-z)_+^j} = \int_0^1
dz\frac{f(z)}{(1-z)_+^j} - \int_0^x dz\frac{f(z)}{(1-z)^j}\\
\\
& =\displaystyle \int_0^1 dz\frac{1}{(1-z)_+^j}\left\{f(z) - \sum_{l=0}^{j-1} \frac{(-1)^l}{l!}
  f^{(l)}(1) (1-z)^l
\right\} - \int_0^x dz\frac{f(z)}{(1-z)^j}\\
\\
& = \displaystyle \int_x^1 dz\frac{1}{(1-z)_+^j}\left\{f(z) - \sum_{l=0}^{j-1} \frac{(-1)^l}{l!}
  f^{(l)}(1) (1-z)^l
\right\} \\
\\
&- \displaystyle \sum_{l=0}^{j-1} \frac{(-1)^l}{l!}
  f^{(l)}(1) \int_0^x\frac{dz}{(1-z)^{j-l}}\,.
\end{array}
\end{equation}
In addition we have that:
\begin{equation}
\int_0^x\frac{dz}{(1-z)^{s}} =
\left\{
\begin{array}{ll}
- \ln(1-x) & \quad s = 1\\
\\
\displaystyle - \frac1{s-1}\left[\frac1{(1-x)^{s-1}}-1\right] & \quad s > 1
\end{array}
\right.\,,
\end{equation}
so that:
\begin{equation}\label{ciao}
\begin{array}{rl}
L_j[f(z)] & = \displaystyle \int_x^1 dz\frac{1}{(1-z)_+^j}\left\{f(z) - \sum_{l=0}^{j-1} \frac{(-1)^l}{l!}
  f^{(l)}(1) (1-z)^l
\right\} \\
\\
& + \displaystyle \sum_{l=0}^{j-2} \frac{(-1)^l}{l!}
  \frac{f^{(l)}(1)}{j-l-1}\left[\frac1{(1-x)^{j-l-1}}-1\right]\\
\\
& + \displaystyle \frac{(-1)^{j-1}}{(j-1)!}f^{(j-1)}(1)\ln(1-x)
\end{array}
\end{equation}
Now, if $f(z)=w_{\alpha}^{(k)}(x_\beta/z)$, that is a Lagrange
interpolation polynomial, the difficult part here is
computing the derivatives $f^{(l)}(1)$. In order to do that, we try
with an indirect strategy: we try to compute the expansion of the
function $w_{\alpha}^{(k)}(x_\beta/z)$ around $z=1$ and then we put
this expansion in relation with the Taylor's expansion to retrieve the
derivatives.

By means of the variable change $x_\beta/z\rightarrow y$, expanding the function $w_{\alpha}^{(k)}(x_\beta/z)$ around $z=1$ is
equivalent to expand the function $w_{\alpha}^{(k)}(y)$ around
$y=x_\beta$ and such an expansion takes the usual form:
\begin{equation}
w_{\alpha}^{(k)}(y) = \sum_{t=0}^\infty \frac{1}{t!} \left[\frac{d^t
  w_\alpha^{(k)}(x_\beta)}{dy^t}\right] (y-x_\beta)^t\,.
\end{equation}
Now, restoring the old variable we find:
\begin{equation}\label{pollo}
w_{\alpha}^{(k)}(x_\beta/z) = \delta_{\alpha\beta}+\sum_{t=1}^\infty \frac{x_\beta^t}{t!} \left[\frac{d^t
  w_\alpha^{(k)}(x_\beta)}{dy^t}\right] \frac{(1-z)^t}{z^t}\,.
\end{equation}
where we have used the fact that $w_{\alpha}^{(k)}(x_\beta) =
\delta_{\alpha\beta}$. This is not all because now we need to expand the function $z^{-t}$
($t \geq 1$) around $z=1$. But one can sho that:
\begin{equation}
\frac{1}{z^t}=\sum_{s=0}^{\infty}\frac{(s+t-1)!}{s!(t-1)!}(1-z)^s\,,
\end{equation}
which can be plugged into eq. (\ref{pollo}), obtaining:
\begin{equation}\label{pollo1}
w_{\alpha}^{(k)}(x_\beta/z) = \delta_{\alpha\beta}+\sum_{s=0}^{\infty}\sum_{t=1}^\infty \frac{(s+t-1)!}{s!(t-1)!}\frac{x_\beta^t}{t!} \left[\frac{d^t
  w_\alpha^{(k)}(x_\beta)}{dy^t}\right] (1-z)^{t+s}\,.
\end{equation}
Now, defining $s+t=l$, we can rearrange the series above as follows:
\begin{equation}\label{pollo2}
w_{\alpha}^{(k)}(x_\beta/z) = \delta_{\alpha\beta}+\sum_{l=1}^{\infty}\frac1{l!}\left\{l!\sum_{t=1}^l \frac{(l-1)!}{(l-t)!(t-1)!}\frac{x_\beta^t}{t!} \left[\frac{d^t
  w_\alpha^{(k)}(x_\beta)}{dy^t}\right] \right\}(1-z)^l
\end{equation}
Now, since the Taylor's expansion of $w_{\alpha}^{(k)}(x_\beta/z)$
around $z=1$ is:
\begin{equation}\label{pollo3}
w_{\alpha}^{(k)}(x_\beta/z) = \delta_{\alpha\beta}+\sum_{l=1}^{\infty}\frac1{l!}\left[\frac{d^l
  w_\alpha^{(k)}(x_\beta/z)}{dz^l}\right]_{z=1} (1-z)^l\,,
\end{equation}
we immediately read that, for $l\geq 1$:
\begin{equation}\label{pollo4}
(f^{(l)}(1)=)\left[\frac{d^l w_\alpha^{(k)}(x_\beta/z)}{dz^l}\right]_{z=1} = l!\sum_{t=1}^l \frac{(l-1)!}{(l-t)!(t-1)!}\frac{x_\beta^t}{t!} \left[\frac{d^t
  w_\alpha^{(k)}(x_\beta)}{dy^t}\right]
\end{equation}

Plugging eq. (\ref{pollo4}) into eq. (\ref{ciao}), we have:
\begin{equation}\label{ciao}
\begin{array}{rl}
L_j[w_\alpha^{(k)}(x_\beta/z)] & = \displaystyle \int_x^1 dz\frac{1}{(1-z)_+^j}\left\{w_\alpha^{(k)}(x_\beta/z) - \sum_{l=0}^{j-1} (-1)^l (1-z)^l
  \sum_{t=1}^l \frac{(l-1)!}{(l-t)!(t-1)!}\frac{x_\beta^t}{t!} \left[\frac{d^t
  w_\alpha^{(k)}(x_\beta)}{dy^t}\right] 
\right\} \\
\\
& + \displaystyle \sum_{l=0}^{j-2} \frac{(-1)^l}{j-l-1}\left[\frac1{(1-x)^{j-l-1}}-1\right]
  \sum_{t=1}^l \frac{(l-1)!}{(l-t)!(t-1)!}\frac{x_\beta^t}{t!} \left[\frac{d^t
  w_\alpha^{(k)}(x_\beta)}{dy^t}\right]\\
\\
& + \displaystyle (-1)^{j-1}\ln(1-x)\sum_{t=1}^{j-1} \frac{(j-2)!}{(j-t-1)!(t-1)!}\frac{x_\beta^t}{t!} \left[\frac{d^t
  w_\alpha^{(k)}(x_\beta)}{dy^t}\right] 
\end{array}
\end{equation}

Finally, we can write that:
\begin{equation}
\begin{array}{rcl}
\displaystyle J_{n,m}[w_\alpha^{(k)}(x_\beta/z)] &=& \displaystyle
\sum_{i=0}^{m-1} (-1)^i {n \choose i}
L_{m-i}[w_\alpha^{(k)}(x_\beta/z)] \\
\\
&+& \displaystyle \sum_{j=0}^{n-m}(-1)^{j+m}{n \choose j+m}\int_x^1dz
\,w_\alpha^{(k)}(x_\beta/z) (1-z)^j\,.
\end{array}
\end{equation}
and:
\begin{equation}
\begin{array}{c}
\displaystyle J_{-1,m}[w_\alpha^{(k)}(x_\beta/z)] = \sum_{i=0}^{m-1} L_{m-i}[w_\alpha^{(k)}(x_\beta/z)] +
\int_x^1dz\frac{w_\alpha^{(k)}(x_\beta/z)}{z}\,.
\end{array}
\end{equation}


\section{Derivatives of the Lagrange Polynomials}\label{DerivativesLP}

Now we need to compute the derivative of the Lagrange polynomial
$w_\alpha^{(k)}(x)$ of degree $k$ on the grid nodes appearing in eq. (\ref{pollo4}). A
Lagrange polynomial solves the following differential system:
\begin{equation}\label{DiffEq}
\left\{
\begin{array}{l}
\displaystyle \frac{d w_\alpha^{(k)}(x)}{dx}
=\left(\sum_{\beta=0\atop \beta\neq\alpha}^k\frac{1}{x-x_\beta}\right)
w_\alpha^{(k)}(x)\\
\\
w_\alpha^{(k)}(x_\alpha) = 1
\end{array}
\right.
 \end{equation}
whose solution is:
\begin{equation}
w_\alpha^{(k)}(x) = \prod_{\beta=0\atop \beta\neq\alpha}^k\frac{x-x_\beta}{x_\alpha-x_\beta}
\end{equation}

Using eq. (\ref{DiffEq}), one can show that the $n$-th derivative of
$w_\alpha^{(k)}(x)$ is equal to zero for $n>k$, while for $n\leq k$:
\begin{equation}\label{NthDerivative}
\frac{d^n w_\alpha^{(k)}(x)}{dx^n}
=\left[\underbrace{\sum_{\beta=0\atop \beta\neq\alpha}^k\frac{1}{x-x_\beta}
\sum_{\gamma=0\atop \gamma\neq\alpha,\beta}^k\frac{1}{x-x_\gamma}
\sum_{\delta=0\atop \delta\neq\alpha,\beta,\gamma}^k\frac{1}{x-x_\delta}
\dots}_{\mbox{$n$ factors}}\right]
w_\alpha^{(k)}(x)
\end{equation}

\subsection{Value of the Derivatives on the Nodes}

The value of the Lagrange interpolation function $w_\alpha^{(k)}(x)$
in correspondence of the nodes is:
\begin{equation}
w_\alpha^{(k)}(x_\rho) = \delta_{\alpha\rho}\,.
\end{equation}
As a consequence, it looks like we could use this equation in
eq. (\ref{NthDerivative}) to get the value of any derivative on the
nodes. Unfortunately this is not so straightforward because, except
for the case $\alpha=\rho$, this expression for the derivatives is
defined only in the limit $x\rightarrow x_\rho$. In particular:
\begin{equation}\label{diagonal}
\frac{d^n w_\alpha^{(k)}(x_\alpha)}{dx^n}
=\underbrace{\sum_{\beta=0\atop \beta\neq\alpha}^k\frac{1}{x_\alpha-x_\beta}
\sum_{\gamma=0\atop \gamma\neq\alpha,\beta}^k\frac{1}{x_\alpha-x_\gamma}
\sum_{\delta=0\atop \delta\neq\alpha,\beta,\gamma}^k\frac{1}{x_\alpha-x_\delta}
\dots}_{\mbox{$n$ factors}}
\end{equation}
while for $\rho\neq\alpha$ we write eq. (\ref{NthDerivative}) in a
different form, that is:
\begin{equation}
\frac{d^n w_\alpha^{(k)}(x)}{dx^n}
=\underbrace{\sum_{\beta=0\atop \beta\neq\alpha}^k\frac{1}{x_\alpha-x_\beta}
\sum_{\gamma=0\atop \gamma\neq\alpha,\beta}^k\frac{1}{x_\alpha-x_\gamma}
\sum_{\delta=0\atop \delta\neq\alpha,\beta,\gamma}^k\frac{1}{x_\alpha-x_\delta}
\dots}_{\mbox{$n$ factors}}
\prod_{\sigma=0\atop\sigma\neq\alpha,\beta,\gamma,\delta,\dots}^k\frac{x - x_\sigma}{x_\alpha-x_\sigma} 
\end{equation}
which can now be evaluate on any node without the need of the limit, yielding:
\begin{equation}\label{offdiagonal}
\frac{d^n w_\alpha^{(k)}(x_\rho)}{dx^n}
=\underbrace{\sum_{\beta=0\atop \beta\neq\alpha}^k\frac{1}{x_\alpha-x_\beta}
\sum_{\gamma=0\atop \gamma\neq\alpha,\beta}^k\frac{1}{x_\alpha-x_\gamma}
\sum_{\delta=0\atop \delta\neq\alpha,\beta,\gamma}^k\frac{1}{x_\alpha-x_\delta}
\dots}_{\mbox{$n$ factors}}
\prod_{\sigma=0\atop\sigma\neq\alpha,\beta,\gamma,\delta,\dots}^k\frac{x_\rho - x_\sigma}{x_\alpha-x_\sigma} 
\end{equation}
which in turn obviously reduces to eq. (\ref{diagonal}) for
$\rho=\alpha$. But for $\rho\neq\alpha$ eq. (\ref{offdiagonal}) can be
further simplified. In fact, the product in eq. (\ref{offdiagonal}) is
different from only if
$\rho = \beta,\gamma,\delta,\dots$ (here by definition we are assuming $\rho \neq \alpha$). Therefore we could write:
\begin{equation}\label{offdiagonal2}
\frac{d^n w_\alpha^{(k)}(x_\rho)}{dx^n}
=\frac{1}{(x_\alpha-x_\rho)^n}
\prod_{\sigma=0\atop\sigma\neq\alpha,\rho}^k\frac{x_\rho - x_\sigma}{x_\alpha-x_\sigma} 
\end{equation}

Eqs. (\ref{diagonal}) and (\ref{offdiagonal}) could be plugged into
eq. (\ref{pollo4}) but, before doing it we can use a
simplification. Suppose that the interpolation grid is such that:
\begin{equation}
x_\alpha - x_\beta = \delta x(\alpha-\beta)
\end{equation}
where $\delta x$ is a constant. Under this assumption, we have that:
\begin{equation}
\frac{d^n w_\alpha^{(k)}(x_\rho)}{dx^n} =
\left\{
\begin{array}{ll}
\displaystyle \frac1{(\delta x)^n}\underbrace{\sum_{\beta=0\atop \beta\neq\alpha}^k\frac{1}{\alpha-\beta}
\sum_{\gamma=0\atop \gamma\neq\alpha,\beta}^k\frac{1}{\alpha-\gamma}
\sum_{\delta=0\atop \delta\neq\alpha,\beta,\gamma}^k\frac{1}{\alpha-\delta}
\dots}_{\mbox{$n$ factors}}  & \quad \rho = \alpha \\
\\
\displaystyle \frac{1}{(\delta x)^n}
\prod_{\sigma=0\atop\sigma\neq\alpha,\rho}^k\frac{\rho - \sigma}{\alpha-\sigma}  & \quad \rho \neq \alpha
\end{array}
\right.
\end{equation}


\section{Implementation in {APFEL}}

In {APFEL} we express PDFs by means of a moving interpolation over
an $x$-space grid, according to the formula:
\begin{equation}
{q}(x,\mu^2)=\sum^{N_{x}}_{\alpha=0}w_{\alpha}^{(k)}(x){q}(x_{\alpha},\mu^2)\,,
\end{equation}
where $w_{\alpha}^{(k)}$ are the "moving" Lagrange interpolation
functions of degree $k$ which take the following form:
\begin{equation}\label{LagrangeFormula}
w_{\alpha}^{(k)}(x) = \sum_{j=0,j \leq \alpha}^{k}\theta(x-x_{\alpha-j})\theta(x_{\alpha-j+1}-x)\prod^{k}_{\delta=0,\delta\ne j}\frac{x-x_{\alpha-j+\delta}}{x_{\alpha}-x_{\alpha-j+\delta}}\,.
\end{equation}


% To implement the tensor gluon evolution in {APFEL} we will need to
% perform integrals of the following kind:
% \begin{equation}
% \begin{array}{rl}
% I &=\displaystyle \int_x^1 dz\frac{z^n w_{\alpha}^{(k)}(x_\beta/z)}{(1-z)_+^m} =
% \int_0^1 dz\frac{z^n w_{\alpha}^{(k)}(x_\beta/z)}{(1-z)_+^m} -
% \int_0^x dz\frac{z^n w_{\alpha}^{(k)}(x_\beta/z)}{(1-z)^m}\\
% \\
% & =\displaystyle \int_0^1 dz\frac{1}{(1-z)_+^m}\left\{z^n
%   w_{\alpha}^{(k)}(x_\beta/z) - \sum_{k=0}^{m-1} \frac{(-1)^k}{k!}
%   \left[\frac{d^k}{dz^k}z^n
%   w_{\alpha}^{(k)}(x_\beta/z) \right]_{z=1} (1-z)^k
% \right\} \\
% \\
% &- \displaystyle \int_0^x dz\frac{z^n
%   w_{\alpha}^{(k)}(x_\beta/z)}{(1-z)^m}\\
% \\
% & = \displaystyle \int_x^1 dz\frac{1}{(1-z)_+^m}\left\{z^n
%   w_{\alpha}^{(k)}(x_\beta/z) - \sum_{k=0}^{m-1} \frac{(-1)^k}{k!}
%   \left[\frac{d^k}{dz^k}z^n
%   w_{\alpha}^{(k)}(x_\beta/z) \right]_{z=1} (1-z)^k
% \right\} \\
% \\
% &- \displaystyle \sum_{k=0}^{m-1} \frac{(-1)^k}{k!}
%   \left[\frac{d^k}{dz^k}z^n
%   w_{\alpha}^{(k)}(x_\beta/z) \right]_{z=1} \int_0^x\frac{dz}{(1-z)^{m-k}}
% \end{array}
% \end{equation}
% This difficult part here is avaluating:
% \begin{equation}
% \left[\frac{d^k}{dz^k}z^n w_{\alpha}^{(k)}(x_\beta/z) \right]_{z=1}
% \end{equation}
% that are essentially the coefficients of the expansion of the function
% $z^n w_{\alpha}^{(k)}(x_\beta/z)$ arount $z=1$. As a consequance we
% may try to expand this function in some clever way: We start noting
% that performing the following change of variable:
% \begin{equation}
% z^n w_{\alpha}^{(k)}(x_\beta/z) \mathop{\longrightarrow}_{x_\beta/z=y}
% \left(\frac{x_\beta}{y}\right)^n w_{\alpha}^{(k)}(y)
% \end{equation}
% we can expand the function arond $y=x_\beta$. But using the geometric series:
% \begin{equation}
% \frac{x_\beta}{y} = \frac{1}{1-\left(1-\frac{y}{x_\beta}\right)} = \sum_{j=0}^\infty \left(1-\frac{y}{x_\beta}\right)^j
% \end{equation}
% Now, considering that:
% \begin{equation}\label{Exp1}
% \begin{array}{c}
% \displaystyle\frac{d^n}{dy^n}\frac{1}{y} = \frac{(-1)^nn!}{y^{n+1}}\Longrightarrow
% \left(\frac{x_\beta}{y}\right)^n =
% \frac{(-1)^{n-1}x_\beta^{n-1}}{(n-1)!}\frac{d^{n-1}}{dy^{n-1}}\frac{x_\beta}{y}\\
% \\
% \displaystyle = \frac{(-1)^{n-1}x_\beta^{n-1}}{(n-1)!}\sum_{j=0}^\infty \frac{d^{n-1}}{dy^{n-1}}\left(1-\frac{y}{x_\beta}\right)^j=\sum_{j=0}^\infty\frac{(j+n-1)!}{j!(n-1)!}\left(1-\frac{y}{x_\beta}\right)^j
% \end{array}
% \end{equation}
% At the same time:
% \begin{equation}
% w_{\alpha}^{(k)}(y) = \sum_{m=0}^k\frac{(-1)^mx_\beta^m}{m!} \left[\frac{d^m w_\alpha^{(k)}(x_\beta)}{dy^m}\right]\left(1-\frac{y}{x_\beta}\right)^m
% \end{equation}
% Therefore:
% \begin{equation}
% \left(\frac{x_\beta}{y}\right)^n w_{\alpha}^{(k)}(y) = \sum_{j=0}^\infty\sum_{m=0}^k\frac{(j+n-1)!}{j!(n-1)!}\frac{(-1)^mx_\beta^m}{m!} \left[\frac{d^m w_\alpha^{(k)}(x_\beta)}{dy^m}\right]\left(1-\frac{y}{x_\beta}\right)^{m+j}
% \end{equation}
% Going back to the variable $z$:
% \begin{equation}
% z^n w_{\alpha}^{(k)}(z/x_\beta) =
% \sum_{j=0}^\infty\sum_{m=0}^k\frac{(j+n-1)!}{j!(n-1)!}\frac{(-1)^jx_\beta^m}{m!}
% \left[\frac{d^m w_\alpha^{(k)}(x_\beta)}{dy^m}\right] \frac{(1-z)^{m+j}}{z^{m+j}}
% \end{equation}
% But using eq. (\ref{Exp1}), we also have that:
% \begin{equation}\label{mimmo}
% \frac{1}{z^{m+j}}=\sum_{t=0}^{\infty}\frac{(t+j+m-1)!}{t!(j+m-1)!}(1-z)^t
% \end{equation}
% Therefore:
% \begin{equation}
% \begin{array}{c}
% \displaystyle z^n w_{\alpha}^{(k)}(z/x_\beta) =
% \sum_{t=0}^{\infty}\sum_{j=0}^\infty\sum_{m=0}^k \frac{(t+j+m-1)!}{t!(j+m-1)!}\frac{(j+n-1)!}{j!(n-1)!}\frac{(-1)^jx_\beta^m}{m!}
% \left[\frac{d^m w_\alpha^{(k)}(x_\beta)}{dy^m}\right] (1-z)^{m+j+t}\\
% \\
% \displaystyle = \sum_{m=0}^k \frac{x_\beta^m}{m!}
% \left[\frac{d^m w_\alpha^{(k)}(x_\beta)}{dy^m}\right]
% \sum_{j=0}^\infty (-1)^j\frac{(j+n-1)!}{j!(n-1)!} \sum_{t=0}^{\infty} \frac{(t+j+m-1)!}{t!(j+m-1)!}(1-z)^{m+j+t}
% \end{array}
% \end{equation}
% What we need here is something of the form:
% \begin{equation}
% \displaystyle z^n w_{\alpha}^{(k)}(z/x_\beta) =\sum_{s=0}^\infty a_s(1-z)^s
% \end{equation}



\newpage

\begin{thebibliography}{alp}

%\cite{Savvidy:2013gsa}
\bibitem{Savvidy:2013gsa} 
  G.~Savvidy,
  %``Invariant scalar product on extended Poincare algebra,''
  J.\ Phys.\ A {\bf 47}, 055204 (2014)
  [arXiv:1308.2695 [hep-th]].
  %%CITATION = ARXIV:1308.2695;%%

\end{thebibliography}

\end{document}
