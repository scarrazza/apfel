%% LyX 2.0.3 created this file.  For more info, see http://www.lyx.org/.
%% Do not edit unless you really know what you are doing.
\documentclass[twoside,english]{paper}
\usepackage{lmodern}
\renewcommand{\ttdefault}{lmodern}
\usepackage[T1]{fontenc}
\usepackage[latin9]{inputenc}
\usepackage[a4paper]{geometry}
\geometry{verbose,tmargin=3cm,bmargin=2.5cm,lmargin=2cm,rmargin=2cm}
\usepackage{color}
\usepackage{babel}
\usepackage{float}
\usepackage{bm}
\usepackage{amsthm}
\usepackage{amsmath}
\usepackage{amssymb}
\usepackage{graphicx}
\usepackage{esint}
\usepackage[unicode=true,pdfusetitle,
 bookmarks=true,bookmarksnumbered=false,bookmarksopen=false,
 breaklinks=false,pdfborder={0 0 0},backref=false,colorlinks=false]
 {hyperref}
\usepackage{breakurl}

\makeatletter

%%%%%%%%%%%%%%%%%%%%%%%%%%%%%% LyX specific LaTeX commands.
%% Because html converters don't know tabularnewline
\providecommand{\tabularnewline}{\\}

%%%%%%%%%%%%%%%%%%%%%%%%%%%%%% Textclass specific LaTeX commands.
\numberwithin{equation}{section}
\numberwithin{figure}{section}

%%%%%%%%%%%%%%%%%%%%%%%%%%%%%% User specified LaTeX commands.
\usepackage{babel}

\@ifundefined{showcaptionsetup}{}{%
 \PassOptionsToPackage{caption=false}{subfig}}
\usepackage{subfig}
\makeatother

\begin{document}

\title{A Common Flavour Basis for the coupled QED$\times$QCD Evolution}

\author{Valerio Bertone$^{a}$}

\institution{$^{a}$PH Department, TH Unit, CERN, CH-1211 Geneva 23, Switzerland}
\maketitle

\begin{abstract}
In this document I will present a suitable flavout basis for the coupled QCD$\times$QED DGLAP evolution of PDFs.
\end{abstract}
\tableofcontents{}

\newpage{}

\section{The Structure of the DGLAP Equation}

The DGLAP equation that governs the PDF evolution has a general
structure that in QCD holds at any perturbative order. Suppose one wants to
study the evolution coupled evolution of the gluon distrubution
function $g(x,\mu)$, the $i$-th quark distribution function
$q_i(x,\mu)$ and the $j$-th anti-quark distribution function
$\overline{q}_j(x,\mu)$. In this case evolution equation would look
like this:
\begin{equation}\label{GeneralDGLAP}
\mu^2\frac{\partial}{\partial \mu^2}
\begin{pmatrix}
q_i\\
g\\
\overline{q}_j
\end{pmatrix} = \sum_{k,l} 
\begin{pmatrix}
P_{q_iq_k} & P_{q_ig} & P_{q_i\overline{q}_l} \\ 
P_{gq_k} & P_{gg} & P_{g\overline{q}_l} \\
P_{\overline{q}_jq_k} & P_{\overline{q}_jg} & P_{\overline{q}_j\overline{q}_l}
\end{pmatrix}
\begin{pmatrix}
q_k\\
g\\
\overline{q}_l
\end{pmatrix}
\end{equation}
where we are understanding the convolution  and where the sum over $k$ and $l$ runs over all $n_f$ the active flavours.
Because of charge conjugation invariance and $SU(n_f)$ flavour
symmetry, one can show that:
\begin{equation}\label{decomposition}
\begin{array}{l}
P_{q_iq_j} = P_{\overline{q}_i\overline{q}_j} = \delta_{ij} P_{qq}^V+P_{qq}^S\\
P_{\overline{q}_iq_j} = P_{q_i\overline{q}_j}  = \delta_{ij} P_{\overline{q}q}^V+P_{\overline{q}q}^S\\
P_{q_ig}  =P_{\overline{q}_ig} = P_{qg} \\
P_{gq_i}  =P_{g\overline{q}_i} = P_{gq} \,.
\end{array}
\end{equation}
Plugging eq. (\ref{decomposition}) into eq. (\ref{GeneralDGLAP}), one finds:
\begin{equation}\label{GeneralDGLAPdec}
\mu^2\frac{\partial}{\partial \mu^2}
\begin{pmatrix}
q_i\\
g\\
\overline{q}_j
\end{pmatrix} =  
\begin{pmatrix}
P_{qq}^V & P_{qg} & P_{q\overline{q}}^V \\ 
P_{gq} & P_{gg} & P_{gq} \\
P_{q\overline{q}}^V & P_{qg} & P_{qq}^V
\end{pmatrix}
\begin{pmatrix}
q_i\\
g\\
\overline{q}_j
\end{pmatrix}+
\begin{pmatrix}
P_{qq}^S & 0 & P_{q\overline{q}}^S \\ 
0 & 0 & 0 \\
P_{q\overline{q}}^S & 0 & P_{qq}^S
\end{pmatrix}
\begin{pmatrix}
\sum_{k}q_k\\
g\\
\sum_l\overline{q}_l
\end{pmatrix}\,.
\end{equation}
Setting $i=j$ and summing and subtracting the first
and the third row/column, we find:
\begin{equation}\label{GeneralDGLAPdecSub1}
\mu^2\frac{\partial}{\partial \mu^2}
\begin{pmatrix}
q_i^+\\
g\\
q_i^-
\end{pmatrix} =  
\begin{pmatrix}
(P_{qq}^V + P_{q\overline{q}}^V) & 2P_{qg} & 0 \\ 
P_{gq} & P_{gg} & 0 \\
0 & 0 & (P_{qq}^V - P_{q\overline{q}}^V)
\end{pmatrix}
\begin{pmatrix}
q_i^+\\
g\\
q_i^-
\end{pmatrix}+
\begin{pmatrix}
(P_{qq}^S + P_{q\overline{q}}^S) & 0  & 0\\ 
0 & 0 & 0 \\
0 & 0 & (P_{qq}^S - P_{q\overline{q}}^S)
\end{pmatrix}
\begin{pmatrix}
\sum_{k}q_k^+\\
g\\
\sum_{k}q_k^-
\end{pmatrix}\,,
\end{equation}
where we have defined:
\begin{equation}
q^\pm_i \equiv q_i \pm \overline{q}_i\,.
\end{equation}
It is evident that in this way we have semi diagonalized the initial
system because now the third equation is decaupled from the rest of
the system. Using the following definitions:
\begin{equation}
\begin{array}{l}
\displaystyle \Sigma \equiv \sum_{k} q_k^+ \\
\\
\displaystyle V \equiv \sum_{k} q_k^- \\
\\
\displaystyle P^\pm \equiv P_{qq}^V \pm P_{q\overline{q}}^V \\
\\
\displaystyle P_{qq} \equiv P^+ + n_f (P_{qq}^S + P_{q\overline{q}}^S)\\
\\
\displaystyle P^V \equiv P^- + n_f (P_{qq}^S - P_{q\overline{q}}^S)
\end{array}\,,
\end{equation}
we have:
\begin{equation}
\left\{
\begin{array}{l}
\displaystyle \mu^2\frac{\partial}{\partial \mu^2}g =P_{gq}g + P_{gg}\Sigma \\
\\
\displaystyle \mu^2\frac{\partial}{\partial \mu^2} q_i^+=
P^+q_i^+ +\frac{1}{n_f}(P_{qq}-P^+)\Sigma + 2P_{qg}g \\ 
\\
\displaystyle \mu^2\frac{\partial}{\partial\mu^2} q_i^- = P^- q_i^-
+\frac{1}{n_f}(P^V-P^-) V
\end{array}
\right.\,.
\end{equation}
At this point we want to generalize this discussion including the QED
corrections. There are two main differences. The first is obviously
the fact that we need to intruduce in the DGLAP equation the parton distribution associated
to the photon $\gamma(x,\mu)$.  The second difference is the fact that the all-order splitting functions no longer undergo to the stringent
simplications of eq. (\ref{decomposition}). In fact, the QED
corrections introduce an asymmetry between down-like quarks ($d$, $s$
and $b$) and top-like quarks ($u$, $c$, $t$), due essentially to the
different electric charge, that breaks the flavour symmetry
for the quark splitting functions. We then must consider the following
extended evolution system:
\begin{equation}\label{EvenMoreGeneralDGLAP}
\mu^2\frac{\partial}{\partial \mu^2}
\begin{pmatrix}
u_j \\
d_i\\
g\\
\gamma\\
\overline{d}_k\\
\overline{u}_h
\end{pmatrix} = \sum_{e,l,m,n} 
\begin{pmatrix}
\mathcal{P}_{u_ju_e} & \mathcal{P}_{u_jd_l} & \mathcal{P}_{u_jg} & \mathcal{P}_{u_j\gamma} & \mathcal{P}_{u_j\overline{d}_m} & \mathcal{P}_{u_j\overline{u}_n} \\ 
\mathcal{P}_{d_iu_e} & \mathcal{P}_{d_id_l} & \mathcal{P}_{d_ig} & \mathcal{P}_{d_i\gamma} & \mathcal{P}_{d_i\overline{d}_m} & \mathcal{P}_{d_i\overline{u}_n} \\ 
\mathcal{P}_{gu_e}  & \mathcal{P}_{gd_l} & \mathcal{P}_{gg} & \mathcal{P}_{g\gamma} & \mathcal{P}_{g\overline{d}_m} & \mathcal{P}_{g\overline{u}_n}\\
\mathcal{P}_{\gamma u_e} & \mathcal{P}_{\gamma d_l} & \mathcal{P}_{\gamma g} & \mathcal{P}_{\gamma\gamma} & \mathcal{P}_{\gamma\overline{d}_m} & \mathcal{P}_{\gamma\overline{u}_n}\\
\mathcal{P}_{\overline{d}_ku_e} & \mathcal{P}_{\overline{d}_kd_l} & \mathcal{P}_{\overline{d}_kg} & \mathcal{P}_{\overline{d}_k\gamma} & \mathcal{P}_{\overline{d}_k\overline{d}_m} & \mathcal{P}_{\overline{d}_k\overline{u}_n}\\
\mathcal{P}_{\overline{u}_hu_e} & \mathcal{P}_{\overline{u}_hd_l} & \mathcal{P}_{\overline{u}_hg} & \mathcal{P}_{\overline{u}_h\gamma} & \mathcal{P}_{\overline{u}_h\overline{d}_m} & \mathcal{P}_{\overline{u}_h\overline{u}_n}
\end{pmatrix}
\begin{pmatrix}
u_e \\
d_l\\
g\\
\gamma\\
\overline{d}_m\\
\overline{u}_n
\end{pmatrix}
\end{equation}
where:
\begin{equation}
u_i= \{u,c,t\}\,,\qquad d_i= \{d,s,b\}\,,\qquad \overline{u}_i= \{\overline{u},\overline{c},\overline{t}\}\,,\qquad \overline{d}_i= \{\overline{d},\overline{s},\overline{b}\}\,.
\end{equation}

Each splitting function in eq. (\ref{EvenMoreGeneralDGLAP}) can be split into two pieces:
\begin{equation}
\mathcal{P}_{ab} = P_{ab} + \widetilde{P}_{ab}\,,
\end{equation}
where $P_{ab}$ is the usual QCD splitting function, $i.e.$ which does
not contain any power of the fine structure constant $\alpha$ and it
only contains powers of the strong coupling $\alpha_s$ and therefore
undergoes to the same simplifications we discussed above. As a further
consequence if $a$ or $b$ is equal to $\gamma$, $P_{ab}$ must vanish.
$\widetilde{P}_{ab}$ instead contains at least one power of $\alpha$. 
In this way we can rearrange eq. (\ref{EvenMoreGeneralDGLAP})  as follows:
\begin{equation}\label{EvenMoreGeneralDGLAPciao}
\begin{array}{rcl}
\displaystyle \mu^2\frac{\partial}{\partial \mu^2}
\begin{pmatrix}
u_j \\
d_i\\
g\\
\gamma\\
\overline{d}_k\\
\overline{u}_h
\end{pmatrix} &=& \displaystyle \sum_{e,l,m,n} 
\left[\begin{pmatrix}
\delta_{je} P_{qq}^V+P_{qq}^S & P_{qq}^S & {P}_{qg} & 0 & P_{q\overline{q}}^S & \delta_{jn} P_{q\overline{q}}^V+P_{q\overline{q}}^S \\ 
P_{qq}^S & \delta_{il} P_{qq}^V+P_{qq}^S & {P}_{qg} & 0 & \delta_{im} P_{q\overline{q}}^V+P_{q\overline{q}}^S & P_{q\overline{q}}^S \\ 
{P}_{gq}  & {P}_{gq} & {P}_{gg} & 0 & {P}_{gq} & {P}_{gq}\\
0 & 0 & 0 & 0 & 0 & 0 \\
P_{q\overline{q}}^S & \delta_{kl} P_{q\overline{q}}^V+P_{q\overline{q}}^S & {P}_{qg} & 0 & \delta_{km} P_{qq}^V+P_{qq}^S & P_{qq}^S\\
\delta_{he} P_{q\overline{q}}^V+P_{q\overline{q}}^S & P_{q\overline{q}}^S & {P}_{qg} & 0 & P_{qq}^S & \delta_{hn} P_{qq}^V+P_{qq}^S
\end{pmatrix}\right.\\
\\
&+&\left.\displaystyle
\begin{pmatrix}
\widetilde{P}_{u_ju_e} & \widetilde{P}_{u_jd_l} & \widetilde{P}_{u_jg} & \widetilde{P}_{u_j\gamma} & \widetilde{P}_{u_j\overline{d}_m} & \widetilde{P}_{u_j\overline{u}_n} \\ 
\widetilde{P}_{d_iu_e} & \widetilde{P}_{d_id_l} & \widetilde{P}_{d_ig} & \widetilde{P}_{d_i\gamma} & \widetilde{P}_{d_i\overline{d}_m} & \widetilde{P}_{d_i\overline{u}_n} \\ 
\widetilde{P}_{gu_e}  & \widetilde{P}_{gd_l} & \widetilde{P}_{gg} & \widetilde{P}_{g\gamma} & \widetilde{P}_{g\overline{d}_m} & \widetilde{P}_{g\overline{u}_n}\\
\widetilde{P}_{\gamma u_e} & \widetilde{P}_{\gamma d_l} & \widetilde{P}_{\gamma g} & \widetilde{P}_{\gamma\gamma} & \widetilde{P}_{\gamma\overline{d}_m} & \widetilde{P}_{\gamma\overline{u}_n}\\
\widetilde{P}_{\overline{d}_ku_e} & \widetilde{P}_{\overline{d}_kd_l} & \widetilde{P}_{\overline{d}_kg} & \widetilde{P}_{\overline{d}_k\gamma} & \widetilde{P}_{\overline{d}_k\overline{d}_m} & \widetilde{P}_{\overline{d}_k\overline{u}_n}\\
\widetilde{P}_{\overline{u}_hu_e} & \widetilde{P}_{\overline{u}_hd_l} & \widetilde{P}_{\overline{u}_hg} & \widetilde{P}_{\overline{u}_h\gamma} & \widetilde{P}_{\overline{u}_h\overline{d}_m} & \widetilde{P}_{\overline{u}_h\overline{u}_n}
\end{pmatrix}\right]
\begin{pmatrix}
u_e \\
d_l\\
g\\
\gamma\\
\overline{d}_m\\
\overline{u}_n
\end{pmatrix}
\end{array}
\end{equation}

Now, since $\alpha$ comes always with an electric charge associated, every
$\widetilde{P}_{ab}$ can factorize out at least an electric charge
$e_u^2=4/9$ or $e_d^2=1/9$. In order to see how this factorization
takes place, we should analyze one by one the splitting functions $\widetilde{P}_{ab}$.

Defining:
\begin{equation}
e_{\Sigma}^2 = e_u^2 n_{u} + e_d^2 n_{d}\,,
\end{equation}
where $n_u$ and $n_d$ are respectively the number of up- and down-type active
quarks such that $n_u + n_d = n_f$, we have that:
\begin{equation}
\begin{array}{cc}
\displaystyle \widetilde{P}_{gg} \rightarrow
e_\Sigma^2\widetilde{P}_{gg}\,, & \displaystyle \widetilde{P}_{g\gamma} \rightarrow e_\Sigma^2\widetilde{P}_{g\gamma}\,,\\
\\
\displaystyle \widetilde{P}_{\gamma g} \rightarrow
e_\Sigma^2\widetilde{P}_{\gamma g}\,, & \displaystyle \widetilde{P}_{\gamma\gamma} \rightarrow e_\Sigma^2\widetilde{P}_{\gamma\gamma}\,.
\end{array}
\end{equation}
This is the consequence of the fact that, having only bosons as
external particles, the presence of any fermion in the splitting must
be summed over all the flavours. This is (should be) true at any
perturbative order.

Now we consider the splitting functions involving one boson and one
quark. Here the situation is more involved because at higer orders it
may happen that the incoming/outcoming quark never couples with a
photon and thus, given that there is at least one power aof $\alpha$,
apart from a term proportional to the charge of the incomin/outcoming
quark there must also be a term proportional to the charge
$e_\Sigma^2$. However, such contributions only appear at three loops
(NNLO) and since here we are only intersted in the two-loop splitting
functions, we have:
\begin{equation}
\begin{array}{cc}
\displaystyle \widetilde{P}_{gu_i} = \widetilde{P}_{g\overline{u}_i} =
e_u^2\widetilde{P}_{gq}\,, & \displaystyle \widetilde{P}_{gd_i} = \widetilde{P}_{g\overline{d}_i} =
e_d^2\widetilde{P}_{gq}\,, \\
\\
\displaystyle \widetilde{P}_{u_i g} = \widetilde{P}_{\overline{u}_i g} =
e_u^2\widetilde{P}_{qg}\,, & \displaystyle \widetilde{P}_{d_i g} =
\widetilde{P}_{\overline{d}_i g} =
e_d^2\widetilde{P}_{qg}\,,\\
\\
\displaystyle \widetilde{P}_{\gamma u_i} = \widetilde{P}_{\gamma \overline{u}_i} =
e_u^2\widetilde{P}_{\gamma q}\,, & \displaystyle \widetilde{P}_{\gamma d_i} = \widetilde{P}_{\gamma \overline{d}_i} =
e_d^2\widetilde{P}_{\gamma q}\,, \\
\\
\displaystyle \widetilde{P}_{u_i \gamma } = \widetilde{P}_{\overline{u}_i \gamma } =
e_u^2\widetilde{P}_{q\gamma }\,, & \displaystyle \widetilde{P}_{d_i \gamma } =
\widetilde{P}_{\overline{d}_i \gamma } =
e_d^2\widetilde{P}_{q\gamma }\,.
\end{array}
\end{equation}

Finally, we consider the splitting functions involving quarks or
anti-quarks in the final and initial states. Again we will limit
ourselves to tow loops and under this resctriction we have:
\begin{equation}\label{decompositionQED}
\begin{array}{l}
\widetilde{P}_{u_iu_j} = \widetilde{P}_{\overline{u}_i\overline{u}_j} = e_u^2\delta_{ij} \widetilde{P}_{qq}^V+e_u^4\widetilde{P}_{qq}^S\\
\widetilde{P}_{d_id_j} = \widetilde{P}_{\overline{d}_i\overline{d}_j} = e_d^2\delta_{ij} \widetilde{P}_{qq}^V+e_d^4\widetilde{P}_{qq}^S\\
\widetilde{P}_{\overline{u}_iu_j} = \widetilde{P}_{u_i\overline{u}_j} = e_u^4\widetilde{P}_{qq}^S\\
\widetilde{P}_{\overline{d}_id_j} = \widetilde{P}_{d_i\overline{d}_j} = e_d^4\widetilde{P}_{qq}^S\\
\widetilde{P}_{u_id_j} = \widetilde{P}_{d_iu_j} = \widetilde{P}_{\overline{u}_id_j} = \widetilde{P}_{d_i\overline{u}_j} = \widetilde{P}_{\overline{d}_iu_j} = \widetilde{P}_{u_i\overline{d}_j} =\widetilde{P}_{\overline{u}_i\overline{d}_j} = \widetilde{P}_{\overline{d}_i\overline{u}_j} = e_u^2e_d^2\widetilde{P}_{qq}^S
\end{array}\,.
\end{equation}

Using the information above, we can now write the QED correction matrix of the splitting functions
up to two loops as follows:
\begin{equation}\label{QEDGen}
\begin{array}{c}
\begin{pmatrix}
e_u^2\delta_{je} \widetilde{P}_{qq}^V+e_u^4\widetilde{P}_{qq}^S & e_u^2e_d^2\widetilde{P}_{qq}^S & e_u^2\widetilde{P}_{qg} & e_u^2\widetilde{P}_{q\gamma} & e_u^2e_d^2\widetilde{P}_{qq}^S & e_u^4\widetilde{P}_{qq}^S \\ 
 e_u^2e_d^2\widetilde{P}_{qq}^S & e_d^2\delta_{il} \widetilde{P}_{qq}^V+e_d^4\widetilde{P}_{qq}^S & e_d^2\widetilde{P}_{qg} & e_d^2\widetilde{P}_{q\gamma} & e_d^4\widetilde{P}_{qq}^S &  e_u^2e_d^2\widetilde{P}_{qq}^S \\ 
e_u^2\widetilde{P}_{gq}  & e_d^2\widetilde{P}_{gq} & e_\Sigma^2\widetilde{P}_{gg} & e_\Sigma^2\widetilde{P}_{g\gamma} & e_d^2\widetilde{P}_{gq} & e_u^2\widetilde{P}_{gq}\\
e_u^2 \widetilde{P}_{\gamma q} & e_d^2\widetilde{P}_{\gamma q} &
e_\Sigma^2\widetilde{P}_{\gamma g} &
e_\Sigma^2\widetilde{P}_{\gamma\gamma} & e_d^2\widetilde{P}_{\gamma q}
& e_u^2\widetilde{P}_{\gamma q}\\
 e_u^2e_d^2\widetilde{P}_{qq}^S & e_d^4\widetilde{P}_{qq}^S & e_d^2\widetilde{P}_{qg} & e_d^2\widetilde{P}_{q\gamma} & e_d^2\delta_{km} \widetilde{P}_{qq}^V+e_d^4\widetilde{P}_{qq}^S &  e_u^2e_d^2\widetilde{P}_{qq}^S\\
e_u^4\widetilde{P}_{qq}^S & e_u^2e_d^2\widetilde{P}_{qq}^S & e_u^2 \widetilde{P}_{qg} & e_u^2\widetilde{P}_{q\gamma} &  e_u^2e_d^2\widetilde{P}_{qq}^S & e_u^2\delta_{hn} \widetilde{P}_{qq}^V+e_u^4\widetilde{P}_{qq}^S
\end{pmatrix}=\\
\\
\begin{pmatrix}
e_u^2\delta_{je}\widetilde{P}_{qq} & 0 & e_u^2\widetilde{P}_{qg} & e_u^2\widetilde{P}_{q\gamma} & 0 & 0 \\ 
0 & e_d^2\delta_{il} \widetilde{P}_{qq}^V & e_d^2\widetilde{P}_{qg} & e_d^2\widetilde{P}_{q\gamma} & 0 & 0 \\ 
e_u^2\widetilde{P}_{gq}  & e_d^2\widetilde{P}_{gq} & e_\Sigma^2\widetilde{P}_{gg} & e_\Sigma^2\widetilde{P}_{g\gamma} & e_d^2\widetilde{P}_{gq} & e_u^2\widetilde{P}_{gq}\\
e_u^2 \widetilde{P}_{\gamma q} & e_d^2\widetilde{P}_{\gamma q} &
e_\Sigma^2\widetilde{P}_{\gamma g} & e_\Sigma^2\widetilde{P}_{\gamma\gamma} & e_d^2\widetilde{P}_{\gamma q}
& e_u^2\widetilde{P}_{\gamma q}\\
0 & 0 & e_d^2\widetilde{P}_{qg} & e_d^2\widetilde{P}_{q\gamma} &
e_d^2\delta_{km} \widetilde{P}_{qq}^V & 0 \\
0 & 0 & e_u^2 \widetilde{P}_{qg} & e_u^2\widetilde{P}_{q\gamma} & 0 & e_u^2\delta_{hn} \widetilde{P}_{qq}^V
\end{pmatrix}+
\begin{pmatrix}
e_u^4\widetilde{P}_{qq}^S & e_u^2e_d^2\widetilde{P}_{qq}^S & 0 & 0 & e_u^2e_d^2\widetilde{P}_{qq}^S & e_u^4\widetilde{P}_{qq}^S \\ 
e_u^2e_d^2\widetilde{P}_{qq}^S & e_d^4\widetilde{P}_{qq}^S & 0 & 0 & e_d^4\widetilde{P}_{qq}^S &  e_u^2e_d^2\widetilde{P}_{qq}^S \\ 
0  & 0 & 0 & 0 & 0 & 0\\
0  & 0 & 0 & 0 & 0 & 0\\
e_u^2e_d^2\widetilde{P}_{qq}^S & e_d^4\widetilde{P}_{qq}^S & 0 & 0 & e_d^4\widetilde{P}_{qq}^S &  e_u^2e_d^2\widetilde{P}_{qq}^S\\
e_u^4\widetilde{P}_{qq}^S & e_u^2e_d^2\widetilde{P}_{qq}^S & 0 & 0 &  e_u^2e_d^2\widetilde{P}_{qq}^S & e_u^4\widetilde{P}_{qq}^S
\end{pmatrix}
\end{array}
\end{equation}

We now apply the same decomposition to the purely QCD matrix, obtaining:
\begin{equation}\label{QCDGen}
\begin{array}{c}
\begin{pmatrix}
\delta_{je} P_{qq}^V+P_{qq}^S & P_{qq}^S & {P}_{qg} & 0 & P_{q\overline{q}}^S & \delta_{jn} P_{q\overline{q}}^V+P_{q\overline{q}}^S \\ 
P_{qq}^S & \delta_{il} P_{qq}^V+P_{qq}^S & {P}_{qg} & 0 & \delta_{im} P_{q\overline{q}}^V+P_{q\overline{q}}^S & P_{q\overline{q}}^S \\ 
{P}_{gq}  & {P}_{gq} & {P}_{gg} & 0 & {P}_{gq} & {P}_{gq}\\
0 & 0 & 0 & 0 & 0 & 0 \\
P_{q\overline{q}}^S & \delta_{kl} P_{q\overline{q}}^V+P_{q\overline{q}}^S & {P}_{qg} & 0 & \delta_{km} P_{qq}^V+P_{qq}^S & P_{qq}^S\\
\delta_{he} P_{q\overline{q}}^V+P_{q\overline{q}}^S & P_{q\overline{q}}^S & {P}_{qg} & 0 & P_{qq}^S & \delta_{hn} P_{qq}^V+P_{qq}^S
\end{pmatrix}=\\
\\
\begin{pmatrix}
\delta_{je} P_{qq}^V & 0 & {P}_{qg} & 0 & \delta_{jm} P_{q\overline{q}}^V & 0 \\ 
0 & \delta_{il} P_{qq}^V & {P}_{qg} & 0 & 0 & \delta_{in} P_{q\overline{q}}^V \\ 
{P}_{gq}  & {P}_{gq} & {P}_{gg} & 0 & {P}_{gq} & {P}_{gq}\\
0 & 0 & 0 & 0 & 0 & 0 \\
0 & \delta_{kl} P_{q\overline{q}}^V & {P}_{qg} & 0 & \delta_{km} P_{qq}^V & 0\\
\delta_{he} P_{q\overline{q}}^V & 0 & {P}_{qg} & 0 & 0 & \delta_{hn} P_{qq}^V
\end{pmatrix} +
\begin{pmatrix}
P_{qq}^S & P_{qq}^S & 0 & 0 & P_{q\overline{q}}^S & P_{q\overline{q}}^S \\ 
P_{qq}^S & P_{qq}^S & 0 & 0 & P_{q\overline{q}}^S & P_{q\overline{q}}^S \\ 
0 & 0 & 0 & 0 & 0 & 0 \\
0 & 0 & 0 & 0 & 0 & 0 \\
P_{q\overline{q}}^S & P_{q\overline{q}}^S & 0 & 0 & P_{qq}^S & P_{qq}^S\\
P_{q\overline{q}}^S & P_{q\overline{q}}^S & 0 & 0 & P_{qq}^S & P_{qq}^S
\end{pmatrix}
\end{array}
\end{equation}

Finally, plugging eqs. (\ref{QEDGen}) and (\ref{QCDGen}) into
eq. (\ref{EvenMoreGeneralDGLAPciao}), performing the sum over $e$,
$l$, $m$ and $n$ and indentifying $k=i$ and $h=j$, we obtain:
\begin{equation}\label{EvenMoreGeneralDGLAPciaociao}
\begin{array}{rcl}
\displaystyle \mu^2\frac{\partial}{\partial \mu^2}
\begin{pmatrix}
u_j \\
d_i\\
g\\
\gamma\\
\overline{d}_i\\
\overline{u}_j
\end{pmatrix} &=& \displaystyle
\left[\begin{pmatrix}
P_{qq}^V & 0 & {P}_{qg} & 0 & 0 & P_{q\overline{q}}^V \\ 
0 & P_{qq}^V & {P}_{qg} & 0 & P_{q\overline{q}}^V & 0 \\ 
{P}_{gq}  & {P}_{gq} & {P}_{gg} & 0 & {P}_{gq} & {P}_{gq}\\
0 & 0 & 0 & 0 & 0 & 0 \\
0 & P_{q\overline{q}}^V & {P}_{qg} & 0 & P_{qq}^V & 0\\
P_{q\overline{q}}^V & 0 & {P}_{qg} & 0 & 0 & P_{qq}^V
\end{pmatrix}+
\begin{pmatrix}
e_u^2\widetilde{P}_{qq} & 0 & e_u^2\widetilde{P}_{qg} & e_u^2\widetilde{P}_{q\gamma} & 0 & 0 \\ 
0 & e_d^2\widetilde{P}_{qq}^V & e_d^2\widetilde{P}_{qg} & e_d^2\widetilde{P}_{q\gamma} & 0 & 0 \\ 
e_u^2\widetilde{P}_{gq}  & e_d^2\widetilde{P}_{gq} & e_\Sigma^2\widetilde{P}_{gg} & e_\Sigma^2\widetilde{P}_{g\gamma} & e_d^2\widetilde{P}_{gq} & e_u^2\widetilde{P}_{gq}\\
e_u^2 \widetilde{P}_{\gamma q} & e_d^2\widetilde{P}_{\gamma q} &
e_\Sigma^2\widetilde{P}_{\gamma g} & e_\Sigma^2\widetilde{P}_{\gamma\gamma} & e_d^2\widetilde{P}_{\gamma q}
& e_u^2\widetilde{P}_{\gamma q}\\
0 & 0 & e_d^2\widetilde{P}_{qg} & e_d^2\widetilde{P}_{q\gamma} &
e_d^2\widetilde{P}_{qq}^V & 0 \\
0 & 0 & e_u^2 \widetilde{P}_{qg} & e_u^2\widetilde{P}_{q\gamma} & 0 & e_u^2\widetilde{P}_{qq}^V
\end{pmatrix}\right]
\begin{pmatrix}
u_j \\
d_i\\
g\\
\gamma\\
\overline{d}_i\\
\overline{u}_j
\end{pmatrix}\\
\\
&+&
\left[
\begin{pmatrix}
P_{qq}^S & P_{qq}^S & 0 & 0 & P_{q\overline{q}}^S & P_{q\overline{q}}^S \\ 
P_{qq}^S & P_{qq}^S & 0 & 0 & P_{q\overline{q}}^S & P_{q\overline{q}}^S \\ 
0 & 0 & 0 & 0 & 0 & 0 \\
0 & 0 & 0 & 0 & 0 & 0 \\
P_{q\overline{q}}^S & P_{q\overline{q}}^S & 0 & 0 & P_{qq}^S & P_{qq}^S\\
P_{q\overline{q}}^S & P_{q\overline{q}}^S & 0 & 0 & P_{qq}^S & P_{qq}^S
\end{pmatrix}+
\begin{pmatrix}
e_u^4\widetilde{P}_{qq}^S & e_u^2e_d^2\widetilde{P}_{qq}^S & 0 & 0 & e_u^2e_d^2\widetilde{P}_{qq}^S & e_u^4\widetilde{P}_{qq}^S \\ 
e_u^2e_d^2\widetilde{P}_{qq}^S & e_d^4\widetilde{P}_{qq}^S & 0 & 0 & e_d^4\widetilde{P}_{qq}^S &  e_u^2e_d^2\widetilde{P}_{qq}^S \\ 
0  & 0 & 0 & 0 & 0 & 0\\
0  & 0 & 0 & 0 & 0 & 0\\
e_u^2e_d^2\widetilde{P}_{qq}^S & e_d^4\widetilde{P}_{qq}^S & 0 & 0 & e_d^4\widetilde{P}_{qq}^S &  e_u^2e_d^2\widetilde{P}_{qq}^S\\
e_u^4\widetilde{P}_{qq}^S & e_u^2e_d^2\widetilde{P}_{qq}^S & 0 & 0 &  e_u^2e_d^2\widetilde{P}_{qq}^S & e_u^4\widetilde{P}_{qq}^S
\end{pmatrix}
\right]
\begin{pmatrix}
\sum_eu_e \\
\sum_ld_l\\
g\\
\gamma\\
\sum_m\overline{d}_m\\
\sum_n\overline{u}_n
\end{pmatrix}
\end{array}
\end{equation}

In order to have the same evolution system in terms of plus- and
minus-distributions, we apply to
eq. (\ref{EvenMoreGeneralDGLAPciaociao}) the following transformation:
\begin{equation}\label{EvenMoreGeneralDGLAPbye}
\mathbf{T} = 
\begin{pmatrix}
1 & 0 & 0 & 0 & 0 & 1 \\
0 & 1 & 0 & 0 & 1 & 0 \\
0 & 0 & 1 & 0 & 0 & 0 \\
0 & 0 & 0 & 1 & 0 & 0 \\
0 & 1 & 0 & 0 & -1 & 0 \\
1 & 0 & 0 & 0 & 0 & -1
\end{pmatrix}\quad\Longrightarrow\quad
\mathbf{T}^{-1} = 
\frac{1}{2}\begin{pmatrix}
1 & 0 & 0 & 0 & 0 & 1 \\
0 & 1 & 0 & 0 & 1 & 0 \\
0 & 0 & 2 & 0 & 0 & 0 \\
0 & 0 & 0 & 2 & 0 & 0 \\
0 & 1 & 0 & 0 & -1 & 0 \\
1 & 0 & 0 & 0 & 0 & -1
\end{pmatrix}
\end{equation}
so that we get:
\begin{equation}
\begin{array}{rcl}
\displaystyle \mu^2\frac{\partial}{\partial \mu^2}
\begin{pmatrix}
u_j^+ \\
d_i^+\\
g\\
\gamma\\
d_i^-\\
u_j^-
\end{pmatrix} &=& \displaystyle
\left[\begin{pmatrix}
P^+ & 0 & 2{P}_{qg} & 0 & 0 & 0 \\ 
0 & P^+ & 2{P}_{qg} & 0 & 0 & 0 \\ 
{P}_{gq}  & {P}_{gq} & {P}_{gg} & 0 & 0 & 0 \\
0 & 0 & 0 & 0 & 0 & 0 \\
0 & 0 & 0 & 0 & P^- & 0\\
0 & 0 & 0 & 0 & 0 & P^-
\end{pmatrix}+
\begin{pmatrix}
e_u^2\widetilde{P}^+ & 0 & 2 e_u^2\widetilde{P}_{qg} & 2 e_u^2\widetilde{P}_{q\gamma} & 0 & 0 \\ 
0 & e_d^2\widetilde{P}^+ & 2 e_d^2\widetilde{P}_{qg} & 2 e_d^2\widetilde{P}_{q\gamma} & 0 & 0 \\ 
e_u^2\widetilde{P}_{gq}  & e_d^2\widetilde{P}_{gq} &
e_\Sigma^2\widetilde{P}_{gg} & e_\Sigma^2\widetilde{P}_{g\gamma} & 0 &0 \\
e_u^2 \widetilde{P}_{\gamma q} & e_d^2\widetilde{P}_{\gamma q} &
e_\Sigma^2\widetilde{P}_{\gamma g} & e_\Sigma^2\widetilde{P}_{\gamma\gamma} & 0 & 0 \\
0 & 0 & 0 & 0 & e_d^2\widetilde{P}^- & 0 \\
0 & 0 & 0 & 0 & 0 & e_u^2\widetilde{P}^-
\end{pmatrix}\right]
\begin{pmatrix}
u_j^+ \\
d_i^+\\
g\\
\gamma\\
d_i^-\\
u_j^-
\end{pmatrix}\\
\\
&+&
\displaystyle \left[
\begin{pmatrix}
P_{qq}-P^+ & P_{qq}-P^+ & 0 & 0 & 0 & 0 \\ 
P_{qq}-P^+ & P_{qq}-P^+ & 0 & 0 & 0 & 0 \\ 
0 & 0 & 0 & 0 & 0 & 0 \\
0 & 0 & 0 & 0 & 0 & 0 \\
0 & 0 & 0 & 0 & P^V-P^- & P^V-P^-\\
0 & 0 & 0 & 0 & P^V-P^- & P^V-P^-
\end{pmatrix}\right.\\
\\
&+& \displaystyle \left.
\begin{pmatrix}
e_u^4(\widetilde{P}_{qq}-\widetilde{P}^+) & e_u^2e_d^2(\widetilde{P}_{qq}-\widetilde{P}^+) & 0 & 0 & 0 & 0 \\ 
e_u^2e_d^2(\widetilde{P}_{qq}-\widetilde{P}^+) & e_d^4(\widetilde{P}_{qq}-\widetilde{P}^+) & 0 & 0 & 0 & 0 \\ 
0  & 0 & 0 & 0 & 0 & 0\\
0  & 0 & 0 & 0 & 0 & 0\\
0  & 0 & 0 & 0 & 0 & 0\\
0  & 0 & 0 & 0 & 0 & 0
\end{pmatrix}
\right]
\frac1{n_f}
\begin{pmatrix}
\sum_eu_e^+ \\
\sum_ld_l^+\\
g\\
\gamma\\
\sum_md_m^-\\
\sum_nu_n^-
\end{pmatrix}
\end{array}
\end{equation}

Using the following definitions:
\begin{equation}
\begin{array}{ll}
\displaystyle \Sigma_u = \sum_{k=i}^{n_u} u_k^+ & \qquad \displaystyle \Sigma_d
= \sum_{k=i}^{n_d} d_k^+ \\
\\
\displaystyle V_u = \sum_{k=i}^{n_u} u_k^- & \qquad \displaystyle V_d
= \sum_{k=i}^{n_d} d_k^-\,,
\end{array}
\end{equation}
which are such that:
\begin{equation}
\Sigma = \Sigma_u + \Sigma_d \quad\mbox{and}\quad V = V_u +  V_d\,,\\
\end{equation}
we can further manipulate eq. (\ref{EvenMoreGeneralDGLAPbye})
obtaining the coupled system:
\begin{equation}
\left\{
\begin{array}{rcl}
\displaystyle \mu^2\frac{\partial g}{\partial \mu^2} &=& (P_{gq} + e_u^2
\widetilde{P}_{gq})\Sigma_u + (P_{gq} + e_d^2
\widetilde{P}_{gq})\Sigma_d+ (P_{gg} + e_\Sigma^2 \widetilde{P}_{gg})g
+ e_\Sigma^2 \widetilde{P}_{g\gamma} \gamma\\
\\
\displaystyle \mu^2\frac{\partial \gamma}{\partial \mu^2} &=& e_u^2
\widetilde{P}_{\gamma q}\Sigma_u + e_d^2
\widetilde{P}_{\gamma q}\Sigma_d+ e_\Sigma^2 \widetilde{P}_{\gamma g}g
+ e_\Sigma^2 \widetilde{P}_{\gamma\gamma} \gamma\\
\\
\displaystyle \mu^2\frac{\partial d_i^+}{\partial \mu^2} &=& (P^+ + e_d^2\widetilde{P}^+)d_i^++ 2(P_{qg} + e_d^2 \widetilde{P}_{qg})g
+ 2e_d^2 \widetilde{P}_{q\gamma} \gamma\\
\\
 &+&\displaystyle \frac1{n_f}[(P_{qq}-P^+)+e_u^2e_d^2(\widetilde{P}_{qq}-\widetilde{P}^+)]\Sigma_u +
\frac1{n_f}[(P_{qq}-P^+)+e_d^4(\widetilde{P}_{qq}-\widetilde{P}^+)]\Sigma_d \\
\\
\displaystyle \mu^2\frac{\partial u_j^+}{\partial \mu^2} &=& (P^+ + e_u^2\widetilde{P}^+)u_j^++ 2(P_{qg} + e_u^2 \widetilde{P}_{qg})g
+ 2e_u^2 \widetilde{P}_{q\gamma} \gamma\\
\\
 &+&\displaystyle \frac1{n_f}[(P_{qq}-P^+)+e_u^4(\widetilde{P}_{qq}-\widetilde{P}^+)]\Sigma_u +
\frac1{n_f}[(P_{qq}-P^+)+e_u^2e_d^2(\widetilde{P}_{qq}-\widetilde{P}^+)]\Sigma_d\\
\\
\displaystyle \mu^2\frac{\partial d_i^-}{\partial \mu^2} &=& \displaystyle  (P^- + e_d^2\widetilde{P}^-)d_i^-+ \frac1{n_f} (P^V-P^-)V_u + \frac1{n_f} (P^V-P^-)V_d \\
\\
\displaystyle \mu^2\frac{\partial u_j^-}{\partial \mu^2} &=& \displaystyle  (P^- +
e_u^2\widetilde{P}^-)u_j^-+ \frac1{n_f} (P^V-P^-)V_u + \frac1{n_f} (P^V-P^-)V_d
\end{array}
\right.\,.
\end{equation}


\section{Evolution Basis}

In order to diagonlize as much as possible the evolution matrix in the
presence of QED corrections avoiding unnecessery couplings between parton distributions, we propose
the following evolution basis:
\begin{equation}
\begin{array}{ll}
\mbox{1) }g & \\
\mbox{2) }\gamma & \\
\mbox{3) }\displaystyle \Sigma = \Sigma_u + \Sigma_d & \quad \mbox{9) }\displaystyle V =V_u +  V_d\\
\mbox{4) }\displaystyle \Delta_\Sigma = \Sigma_u - \Sigma_d & \quad \mbox{10) }\displaystyle \Delta_V = V_u - V_d\\
\mbox{5) }T_1^u = u^+ - c^+ &\quad \mbox{11) }V_1^u = u^- - c^- \\
\mbox{6) }T_2^u = u^+ + c^+ - 2t^+ &\quad \mbox{12) }V_2^u = u^- + c^- - 2t^-\\
\mbox{7) }T_1^d = d^+ - s^+ &\quad \mbox{13) }V_1^d = d^- - s^- \\
\mbox{8) }T_2^d = d^+ + s^+ - 2b^+ &\quad \mbox{14) }V_2^d = d^- + s^- - 2b^-
\end{array}
\end{equation}

In this basis the evolution system becomes:
\begin{equation}\label{pippo}
\begin{array}{l}
\left\{
\begin{array}{rcl}
\displaystyle \mu^2\frac{\partial g}{\partial \mu^2} &=& \displaystyle
(P_{gq} + \eta^+
\widetilde{P}_{gq})\Sigma + \eta^-
\widetilde{P}_{gq}\Delta_\Sigma+ (P_{gg} + e_\Sigma^2 \widetilde{P}_{gg})g
+ e_\Sigma^2 \widetilde{P}_{g\gamma} \gamma\\
\\
\displaystyle \mu^2\frac{\partial \gamma}{\partial \mu^2} &=& \eta^+\widetilde{P}_{\gamma q}\Sigma + \eta^-\widetilde{P}_{\gamma q}\Delta_\Sigma+ e_\Sigma^2 \widetilde{P}_{\gamma g}g
+ e_\Sigma^2 \widetilde{P}_{\gamma\gamma} \gamma\\
\\
\displaystyle \mu^2\frac{\partial \Sigma}{\partial \mu^2} &=&
\displaystyle 
\left[P_{qq} + \eta^+\widetilde{P}^+
  +\frac{\eta^+e_\Sigma^2}{n_f}(\widetilde{P}_{qq} -
  \widetilde{P}^+)\right]\Sigma+ \left[\eta^-\widetilde{P}^++\frac{\eta^-e_\Sigma^2}{n_f}(\widetilde{P}_{qq} -
  \widetilde{P}^+)\right]\Delta_\Sigma\\
\\
 &+&  2(n_fP_{qg} + e_\Sigma^2 \widetilde{P}_{qg})g
+ 2e_\Sigma^2 \widetilde{P}_{q\gamma} \gamma\\
\\
\displaystyle \mu^2\frac{\partial \Delta_\Sigma}{\partial \mu^2} &=&\displaystyle
\left[\eta^-\widetilde{P}^++\frac{n_u-n_d}{n_f}(P_{qq}-P^+)+\frac{\eta^+\delta_e^2}{n_f}(\widetilde{P}_{qq}-\widetilde{P}^+)\right]\Sigma + \left[P^+ +
\eta^+\widetilde{P}^++\frac{\eta^-\delta_e^2}{n_f}(\widetilde{P}_{qq}-\widetilde{P}^+)\right]\Delta_\Sigma\\
\\
&+& 2[(n_u-n_d)P_{qg} + \delta_e^2 \widetilde{P}_{qg}]g
+ 2\delta_e^2 \widetilde{P}_{q\gamma} \gamma\\
\end{array}
\right.
\\
\\
\begin{array}{rcl}
\\
\displaystyle \mu^2\frac{\partial T^u_{1,2}}{\partial \mu^2} &=&
\displaystyle (P^+ + e_u^2\widetilde{P}^+) T^u_{1,2}\\
\\
\displaystyle \mu^2\frac{\partial T^d_{1,2}}{\partial \mu^2} &=&
\displaystyle (P^+ + e_d^2\widetilde{P}^+) T^d_{1,2}
\end{array}
\\
\\
\left\{
\begin{array}{rcl}
\displaystyle \mu^2\frac{\partial V}{\partial \mu^2} &=& \displaystyle
(P^V+\eta^+\widetilde{P}^- ) V + \eta^- \widetilde{P}^-\Delta_V\\
\\
\displaystyle \mu^2\frac{\partial \Delta_V}{\partial \mu^2} &=&
\displaystyle  \left[\frac{n_u-n_d}{n_f}(P^V-P^-) +\eta^-\widetilde{P}^- \right]V + \left[P^-+
  \eta^+\widetilde{P}^-\right]\Delta_V
\end{array}
\right.
\\
\\
\begin{array}{rcl}
\displaystyle \mu^2\frac{\partial V^u_{1,2}}{\partial \mu^2} &=&
\displaystyle (P^- + e_u^2\widetilde{P}^-) V^u_{1,2}\\
\\
\displaystyle \mu^2\frac{\partial V^d_{1,2}}{\partial \mu^2} &=&
\displaystyle (P^- + e_d^2\widetilde{P}^-) V^d_{1,2}
\end{array}
\end{array}
\end{equation}
with the definition:
\begin{equation}
\delta_e^2 = n_u e_u^2 -n_d e_d^2
\end{equation}
and where we have used the curly bracket to denote the coupled
equations. The main thing to notice here is the fact that there are
two coupled sub-system. This is in contrast with what we had in pure
QCD where there was only one coupled system.

Now, let's write the eq. (\ref{pippo}) in a matricial form, separating the
pure QCD splitting functions (those without tilde) from the QED
contributions:
\begin{equation}
\begin{array}{rcl}
\displaystyle\mu^2\frac{\partial}{\partial \mu^2}
\begin{pmatrix}
g\\
\gamma\\
\Sigma\\
\Delta_\Sigma
\end{pmatrix} &=& \displaystyle \left[
\begin{pmatrix}
P_{gg} & 0 & P_{gq} & 0 \\
0 & 0 & 0 & 0 \\
2n_fP_{qg} & 0 & P_{qq} & 0 \\
\frac{n_u-n_d}{n_f} 2n_fP_{qg} & 0 & \frac{n_u-n_d}{n_f}(P_{qq}-P^+) & P^+
\end{pmatrix}\right.
\\
\\
&+&\left.\begin{pmatrix}
e_\Sigma^2 \widetilde{P}_{gg}          & e_\Sigma^2 \widetilde{P}_{g\gamma} & \eta^+\widetilde{P}_{gq} & \eta^-\widetilde{P}_{gq} \\
e_\Sigma^2 \widetilde{P}_{\gamma g} & e_\Sigma^2 \widetilde{P}_{\gamma\gamma} & \eta^+\widetilde{P}_{\gamma q} &\eta^-\widetilde{P}_{\gamma q} \\
2 e_\Sigma^2 \widetilde{P}_{qg} & 2 e_\Sigma^2 \widetilde{P}_{q\gamma}
& \eta^+\widetilde{P}^++\frac{\eta^+e_\Sigma^2}{n_f}(\widetilde{P}_{qq}-\widetilde{P}^+)  & \eta^-\widetilde{P}^++\frac{\eta^-e_\Sigma^2}{n_f}(\widetilde{P}_{qq}-\widetilde{P}^+)\\
2 \delta_e^2 \widetilde{P}_{qg} & 2 \delta_e^2 \widetilde{P}_{q\gamma}
&\eta^-\widetilde{P}^++\frac{\eta^+\delta_e^2}{n_f}(\widetilde{P}_{qq}-\widetilde{P}^+)
&\eta^+\widetilde{P}^++\frac{\eta^-\delta_e^2}{n_f}(\widetilde{P}_{qq}-\widetilde{P}^+)
\end{pmatrix}
\right]
\begin{pmatrix}
g\\
\gamma\\
\Sigma\\
\Delta_\Sigma
\end{pmatrix}
\end{array}
\end{equation}


\begin{equation}
\displaystyle\mu^2\frac{\partial}{\partial \mu^2}
\begin{pmatrix}
V\\
\Delta_V
\end{pmatrix} = 
\left[
\begin{pmatrix}
P^V & 0 \\
\frac{n_u-n_d}{n_f}(P^V-P^-)  & P^-
\end{pmatrix}
+
\begin{pmatrix}
\eta^+\widetilde{P}^- & \eta^-\widetilde{P}^- \\
\eta^-\widetilde{P}^- & \eta^+\widetilde{P}^- 
\end{pmatrix}
\right]
\begin{pmatrix}
V\\
\Delta_V
\end{pmatrix}
\end{equation}

It should finally be said that, every time one of one quark flavour is not
active, the non-singlet distributions $T_{1,2}^{u,d}$ and
$V_{1,2}^{u,d}$ involving that quark flavour, starts evolving as a
singlet distribution according to the following equations:
\begin{equation}
\begin{array}{l}
\displaystyle T_{1,2}^{u} = \frac{\Sigma+\Delta_\Sigma}{2}\,,\\
\\
\displaystyle T_{1,2}^{d} = \frac{\Sigma-\Delta_\Sigma}{2}\,,\\
\\
\displaystyle V_{1,2}^{u} = \frac{V+\Delta_V}{2}\,,\\
\\
\displaystyle V_{1,2}^{d} = \frac{V-\Delta_V}{2}\,.
\end{array}
\end{equation}


\section{QED corrections at LO}

If we consider only LO QED corrections to the PDF evolution equations,
there are a few simplifications that make the evolution sistem
simpler. In particular we have that:
\begin{equation}
\widetilde{P}_{gg} = \widetilde{P}_{g\gamma} = \widetilde{P}_{\gamma
  g} = \widetilde{P}_{gq} = \widetilde{P}_{qg} = 0\,.
\end{equation}
In addition:
\begin{equation}
\widetilde{P}^+ = \widetilde{P}^- = \widetilde{P}_{qq}\,.
\end{equation}

With these simplifications we can rewrite the above evolution systems
as follows:

\begin{equation}\label{APFELsys}
\begin{array}{rcl}
\displaystyle\mu^2\frac{\partial}{\partial \mu^2}
\begin{pmatrix}
g\\
\gamma\\
\Sigma\\
\Delta_\Sigma
\end{pmatrix} &=& \displaystyle \left[
\begin{pmatrix}
P_{gg} & 0 & P_{gq} & 0 \\
0 & 0 & 0 & 0 \\
2n_fP_{qg} & 0 & P_{qq} & 0 \\
\frac{n_u-n_d}{n_f} 2n_fP_{qg} & 0 & \frac{n_u-n_d}{n_f}(P_{qq}-P^+) & P^+
\end{pmatrix}\right.
\\
\\
&+&\left.\begin{pmatrix}
0 & 0 & 0 & 0 \\
0 & e_\Sigma^2 \widetilde{P}_{\gamma\gamma} & \eta^+\widetilde{P}_{\gamma q} &\eta^-\widetilde{P}_{\gamma q} \\
0 & 2 e_\Sigma^2 \widetilde{P}_{q\gamma} & \eta^+\widetilde{P}_{qq} & \eta^-\widetilde{P}_{qq}\\
0 & 2 \delta_e^2 \widetilde{P}_{q\gamma}
&\eta^-\widetilde{P}_{qq}
&\eta^+\widetilde{P}_{qq}
\end{pmatrix}
\right]
\begin{pmatrix}
g\\
\gamma\\
\Sigma\\
\Delta_\Sigma
\end{pmatrix}
\end{array}
\end{equation}


\begin{equation}
\displaystyle\mu^2\frac{\partial}{\partial \mu^2}
\begin{pmatrix}
V\\
\Delta_V
\end{pmatrix} = 
\left[
\begin{pmatrix}
P^V & 0 \\
\frac{n_u-n_d}{n_f}(P^V-P^-)  & P^- 
\end{pmatrix}
+
\begin{pmatrix}
\eta^+\widetilde{P}_{qq} & \eta^-\widetilde{P}_{qq} \\
\eta^-\widetilde{P}_{qq} & \eta^+\widetilde{P}_{qq}
\end{pmatrix}
\right]
\begin{pmatrix}
V\\
\Delta_V
\end{pmatrix}
\end{equation}



\begin{equation}
\begin{array}{l}
\begin{array}{rcl}
\\
\displaystyle \mu^2\frac{\partial T^u_{1,2}}{\partial \mu^2} &=&
\displaystyle (P^+ + e_u^2\widetilde{P}_{qq}) T^u_{1,2}\\
\\
\displaystyle \mu^2\frac{\partial T^d_{1,2}}{\partial \mu^2} &=&
\displaystyle (P^+ + e_d^2\widetilde{P}_{qq}) T^d_{1,2}
\end{array}
\\
\\
\begin{array}{rcl}
\displaystyle \mu^2\frac{\partial V^u_{1,2}}{\partial \mu^2} &=&
\displaystyle (P^- + e_u^2\widetilde{P}_{qq}) V^u_{1,2}\\
\\
\displaystyle \mu^2\frac{\partial V^d_{1,2}}{\partial \mu^2} &=&
\displaystyle (P^- + e_d^2\widetilde{P}_{qq}) V^d_{1,2}
\end{array}
\end{array}
\end{equation}

Notice that eq. (\ref{APFELsys}), recognizing that
$2e_\Sigma^2=\theta^-$ and $2\delta_e^2=\theta^+$ (up to a factor $N_c$), is
consistent with eq. (9) of the APFEL paper.



\bibliographystyle{unsrt}
\phantomsection\addcontentsline{toc}{section}{\refname}\nocite{*}
\bibliography{bibliography}

\end{document}
