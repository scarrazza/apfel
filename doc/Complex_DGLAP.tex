%% LyX 2.0.2 created this file.  For more info, see http://www.lyx.org/.
%% Do not edit unless you really know what you are doing.
\documentclass[twoside,english]{article}
\usepackage{lmodern}
\usepackage[T1]{fontenc}
\usepackage[latin9]{inputenc}
\usepackage[a4paper]{geometry}
\geometry{verbose,tmargin=3cm,bmargin=2.5cm,lmargin=2cm,rmargin=2cm}
\usepackage{babel}
\usepackage{float}
\usepackage{bm}
\usepackage{amsmath}
\usepackage{esint}
\usepackage[unicode=true,pdfusetitle,
 bookmarks=true,bookmarksnumbered=false,bookmarksopen=false,
 breaklinks=false,pdfborder={0 0 0},backref=false,colorlinks=false]
 {hyperref}

\makeatletter

%%%%%%%%%%%%%%%%%%%%%%%%%%%%%% LyX specific LaTeX commands.
%% Because html converters don't know tabularnewline
\providecommand{\tabularnewline}{\\}

%%%%%%%%%%%%%%%%%%%%%%%%%%%%%% Textclass specific LaTeX commands.
%\numberwithin{equation}{section}

\makeatother

\begin{document}

\title{DGLAP Evolution Equation on the Complex Plane}

\author{Valerio Bertone}
\maketitle
\begin{abstract}
In this document I describe the extension of the DGLAP equation for
complex values of the factorization scale $\mu$ and the relative
implementation in {\tt APFEL}.
\end{abstract}

\section{DGLAP on the Complex Plane}

Let us start from the usual DGLAP evolution equation for the
distribution $f$(\footnote{At this stage it is not necessary to
  distinguish between singlet or non-singlet distributions. We will
  consider these cases separately later once the formalism has been
  settled.}):
\begin{equation}\label{DGLAPStandard}
\mu^2\frac{\partial f}{\partial \mu^2} =
P\left(x,\alpha_s(\mu)\right)\otimes f(x,\mu)\,
\end{equation}
where $\otimes$ represents the usual Mellin convolution such that:
\begin{equation}
A(x)\otimes B(x) \equiv \int_0^1dz \int_0^1dy A(y)B(z)\delta(x-yz)=
\int^1_x\frac{dy}{y}
A(x)B\left(\frac{y}{x}\right) = \int^1_x\frac{dy}{y}
A\left(\frac{y}{x}\right)B(x)\,.
\end{equation}
Now we consider the RGE for the strong coupling $\alpha_s$, that reads:
\begin{equation}\label{RGEalphas}
\mu^2\frac{\partial \alpha_s}{\partial \mu^2} = \beta(\alpha_s)\,,
\end{equation}
and combine it with the DGLAP equation in eq.~(\ref{DGLAPStandard}),
obtaining the DGLAP equation differential in $\alpha_s$:
\begin{equation}\label{DGLAPalphas}
\frac{\partial f}{\partial \alpha_s} =
R\left(x,\alpha_s\right)\otimes f(x,\alpha_s)\,
\end{equation}
where:
\begin{equation}
R\left(x,\alpha_s\right) = \frac{P\left(x,\alpha_s\right)}{\beta(\alpha_s)}\,.
\end{equation}

The next fundamental step is the promotion of the factorization scale
$\mu$ from a real to a complex variable:
\begin{equation}
\mu \rightarrow \eta = \mu + i\nu\,.
\end{equation}
As a consequence, we need to promote also the strong the PDF $f$ and
the strong coupling $\alpha_s$ to being complex functions, that is:
\begin{equation}
\begin{array}{rcl}
f &\rightarrow& F = f + ig\,,\\
\alpha_s &\rightarrow& \zeta_s = \alpha_s + i\xi_s\,.
\end{array}
\end{equation}
This has as a further consequence that the DGLAP and the $\alpha_s$
evolution equations in eqs.~(\ref{DGLAPStandard})
and~(\ref{RGEalphas}) become complex differential equations:
\begin{equation}
\eta^2\frac{\partial F}{\partial \eta^2} =
P\left(x,\zeta_s(\eta)\right)\otimes F(x,\eta)\,,
\end{equation}
and:
\begin{equation}
\eta^2\frac{\partial \zeta_s}{\partial \eta^2} = \beta(\zeta_s)\,,
\end{equation}
that can be again combined in:
\begin{equation}\label{DGLAPalphasComplex}
\frac{\partial F}{\partial \zeta_s} =
R\left(x,\zeta_s\right)\otimes F(x,\zeta_s)\,.
\end{equation}

The main goal is the solution of eq.~(\ref{DGLAPalphasComplex}). The
starting observation is the fact that the complex function $F$ must be
an analytical function of the complex variable $\zeta_s$. This implies
that the real and the complex parts of $F$ must obey the
Cauchy-Riemann equations, that is:
\begin{equation}
\begin{array}{rcl}
\displaystyle \frac{\partial f}{\partial \alpha_s} & = & \displaystyle
                                                         \frac{\partial
                                                         g}{\partial
                                                         \xi_s} \\
\\
\displaystyle \frac{\partial f}{\partial \xi_s} & = & \displaystyle
                                                         -\frac{\partial
                                                         g}{\partial
                                                         \alpha_s}
\end{array}\,,
\end{equation}
so that the derivative of $F$ with respect to $\eta_s$ can be expanded
as:
\begin{equation}\label{ComplexDerivative}
\frac{\partial F}{\partial \zeta_s} = \frac{\partial f}{\partial
  \alpha_s} + i \frac{\partial g}{\partial \alpha_s} =\frac{\partial f}{\partial
  \xi_s} - i \frac{\partial g}{\partial \xi_s}\,.
\end{equation}
Now let us consider the function $R$. Being it a complex function, it
can be split into a real and a complex part:
\begin{equation}
R = S + i T\,,
\end{equation}
and thus:
\begin{equation}\label{Rexp}
R\otimes F = ( S + i T )\otimes ( f+i g ) = ( S \otimes f - T \otimes g )
+ i ( T \otimes f + S \otimes g )\,.
\end{equation}

We can now combine eqs.~(\ref{ComplexDerivative}) and~(\ref{Rexp})
into eq.~(\ref{DGLAPalphasComplex}). This allows us to obtain two sets
of coupled real differential equations that can be written in the following
matricial form:
\begin{equation}\label{DGLAPcomplex1}
\frac{\partial }{\partial \alpha_s}{f \choose g} = 
\begin{pmatrix}
S & T \\
T & -S
\end{pmatrix}
\otimes {f \choose g}\,,
\end{equation}
and:
\begin{equation}\label{DGLAPcomplex2}
\frac{\partial }{\partial \xi_s}{f \choose g} = 
\begin{pmatrix}
-T & -S \\
S & -T
\end{pmatrix}
\otimes {f \choose g}\,.
\end{equation}
The solution of eqs.~(\ref{DGLAPcomplex1}) and~(\ref{DGLAPcomplex2})
allows one to obtain the dependence of the real functions $f$ and $g$
($i.e.$ the real and the complex part of the ``complex'' PDF $F$) on
the real variables $\alpha_s$ and $\xi_s$ ($i.e.$ the real and the
complex part of the ``complex'' strong coupling $\zeta_s$) which in turn
are functions of the complex factorization scale $\eta$. 

Now we need to extract the functions $S$ and $T$ from $R$ and in the
next section we will show their form at leading order (LO) in QCD.

\subsection{Solution at LO}

At LO in QCD we have that:
\begin{equation}
R\left(x,\zeta_s\right) =
\frac{P\left(x,\zeta_s\right)}{\beta(\zeta_s)} =
-\frac{P^{(0)}(x)}{\beta_0} \frac{1}{\zeta_s} = -\frac{P^{(0)}(x)}{\beta_0} \frac{\alpha_s-i\xi_s}{\alpha^2_s+\xi_s^2}
\end{equation}
and thus:
\begin{equation}\label{LOKernels}
S(x,\alpha_s,\xi_s) = -\frac{P^{(0)}(x)}{\beta_0}
\frac{\alpha_s}{\alpha^2_s+\xi_s^2}\quad\mbox{and}\quad T(x,\alpha_s,\xi_s) = \frac{P^{(0)}(x)}{\beta_0} \frac{\xi_s}{\alpha^2_s+\xi_s^2}\,.
\end{equation}
Using eq.~(\ref{LOKernels}), eqs.~(\ref{DGLAPcomplex1})
and~(\ref{DGLAPcomplex2}) become:
\begin{equation}\label{DGLAPcomplex1LO}
\frac{\partial }{\partial \alpha_s}{f \choose g} = 
-\frac{P^{(0)}(x)}{\beta_0}
\frac{1}{\alpha^2_s+\xi_s^2} \begin{pmatrix}
\alpha_s & -\xi_s \\
-\xi_s & -\alpha_s
\end{pmatrix}
\otimes {f \choose g}\,,
\end{equation}
and:
\begin{equation}\label{DGLAPcomplex2LO}
\frac{\partial }{\partial \xi_s}{f \choose g} = 
-\frac{P^{(0)}(x)}{\beta_0}
\frac{1}{\alpha^2_s+\xi_s^2} \begin{pmatrix}
\xi_s & -\alpha_s \\
\alpha_s & \xi_s
\end{pmatrix}
\otimes {f \choose g}\,.
\end{equation}
In Mellin space, the Mellin convolution of the equations above becomes
a simple product and can then be solved more easily. In fact, the
equations above in Mellin space become:
\begin{equation}\label{DGLAPcomplex1LOMellin}
\frac{\partial }{\partial \alpha_s}{f \choose g} = 
-\frac{\gamma^{(0)}(N)}{\beta_0}
\frac{1}{\alpha^2_s+\xi_s^2} \begin{pmatrix}
\alpha_s & -\xi_s \\
-\xi_s & -\alpha_s
\end{pmatrix}
{f \choose g}\,,
\end{equation}
and:
\begin{equation}\label{DGLAPcomplex2LOMellin}
\frac{\partial }{\partial \xi_s}{f \choose g} = 
-\frac{\gamma^{(0)}(N)}{\beta_0}
\frac{1}{\alpha^2_s+\xi_s^2} \begin{pmatrix}
\xi_s & -\alpha_s \\
\alpha_s & \xi_s
\end{pmatrix}
{f \choose g}\,.
\end{equation}
Defining:
\begin{equation}
\mathbf{F} \equiv {f \choose g}\quad\mbox{and}\quad R_0 = \frac{\gamma^{(0)}(N)}{\beta_0}\,,
\end{equation}
and conidering that:
\begin{equation}
\int dx\frac{x}{x^2+y^2} = \frac12\ln(x^2+y^2) \quad\mbox{and}\quad
\int dx\frac{y}{x^2+y^2} = \mbox{atan}\left(\frac{x}{y}\right)\,,
\end{equation}
the solution of eqs.~(\ref{DGLAPcomplex1LOMellin})
and~(\ref{DGLAPcomplex2LOMellin}) is:
\begin{equation}\label{SolutionLO1}
\mathbf{F}(N,\alpha_s,\xi_s) = \mathbf{F}(N,\alpha_{s,0},\xi_s)\exp\left[-R_0\Gamma_\alpha\right]\,,
\end{equation}
and:
\begin{equation}\label{SolutionLO2}
\mathbf{F}(N,\alpha_s,\xi_s) = \mathbf{F}(N,\alpha_{s},\xi_{s,0})\exp\left[-R_0\Gamma_\xi\right]\,,
\end{equation}
with:
\begin{equation}
\Gamma_\alpha=
\begin{pmatrix}
\frac12\ln\left(\frac{\alpha_s^2+\xi_s^2}{\alpha_{s,0}^2+\xi_s^2}\right) &
-\mbox{atan}\left(\frac{\alpha_s}{\xi_s}\right) +
\mbox{atan}\left(\frac{\alpha_{s,0}}{\xi_s}\right) \\
-\mbox{atan}\left(\frac{\alpha_s}{\xi_s}\right) +
\mbox{atan}\left(\frac{\alpha_{s,0}}{\xi_s}\right) & -\frac12\ln\left(\frac{\alpha_s^2+\xi_s^2}{\alpha_{s,0}^2+\xi_s^2}\right)
\end{pmatrix}\,,
\end{equation}
and:
\begin{equation}
\Gamma_\xi=
\begin{pmatrix}
\frac12\ln\left(\frac{\alpha_s^2+\xi_s^2}{\alpha_{s}^2+\xi_{s,0}^2}\right) &
-\mbox{atan}\left(\frac {\xi_s}{\alpha_s}\right) +
\mbox{atan}\left(\frac {\xi_{s,0}}{\alpha_{s}}\right) \\
\mbox{atan}\left(\frac{\xi_s}{\alpha_s}\right) -
\mbox{atan}\left(\frac{\xi_{s,0}}{\alpha_{s}}\right) & \frac12\ln\left(\frac{\alpha_s^2+\xi_s^2}{\alpha_{s}^2+\xi_{s,0}^2}\right)
\end{pmatrix}\,.
\end{equation}
It is interesting to observe that, if the complex strong coupling
$\zeta_s$ becomes real, $i.e.$ $\xi_s\rightarrow0$, the matrices above become:
\begin{equation}
\lim_{\xi_s\rightarrow0}\Gamma_\alpha= \ln\left(\frac{\alpha_s}{\alpha_{s,0}}\right)
\begin{pmatrix}
1 & 0 \\
0 & -1
\end{pmatrix}\,,
\end{equation}
and:
\begin{equation}
\lim_{\xi_s\rightarrow0} \Gamma_\xi= 0\,,
\end{equation}
and thus the evolution in $\alpha_s$ of the real part of the PDF $f$
reduces to the expected one while no evolution in $\xi_s$ is left. As
for the evolution in $\alpha_s$ of the imaginary part of the PDF $g$,
there is an evolution factor but it is decoupled from the real part
and thus it has an effect only if the imaginary part of the initial
scale PDF is different from zero.

The solutions in eqs.~(\ref{SolutionLO1}) and~(\ref{SolutionLO2}) are
written in a pretty formal way because they imply the exponential of
matrices. However, since $\Gamma_\alpha$ and $\Gamma_\xi$ are 2 by 2
matrices, their exponential in known a simple closed form. In
particular:
\begin{equation}\label{ExpMatrix}
\exp\begin{pmatrix}
a & b \\
c & d
\end{pmatrix} = \frac{\exp\left[(a+d)/2\right]}{\Delta}
\begin{pmatrix}
m_{11} & m_{12} \\
m_{21} & m_{22}
\end{pmatrix}
\end{equation}
where:
\begin{equation}
\Delta = \sqrt{(a-d)^2+4bc}
\end{equation}
and:
\begin{equation}\label{ExpMatrixEntries}
\begin{array}{rcl}
m_{11} & = & \displaystyle \Delta\mbox{cosh}\left(\frac{\Delta}2\right)+(a-d)\mbox{sinh}\left(\frac{\Delta}2\right)\\
\\
m_{12} & = & \displaystyle 2 b\,\mbox{sinh}\left(\frac{\Delta}2\right)\\
\\
m_{21} & = & \displaystyle 2 c\,\mbox{sinh}\left(\frac{\Delta}2\right)\\
\\
m_{22} & = & \displaystyle \Delta\mbox{cosh}\left(\frac{\Delta}2\right)-(a-d)\mbox{sinh}\left(\frac{\Delta}2\right)
\end{array}
\end{equation}
Given the structure of $\Gamma_\alpha$ and $\Gamma_\xi$, we can
simplify the formulas above. In the case of $\Gamma_\alpha$ the
structure is:
\begin{equation}
\Gamma_\alpha = \begin{pmatrix}
a & -b \\
-b & -a
\end{pmatrix}
\end{equation}
with:
\begin{equation}
\begin{array}{rcl}
a &=&\displaystyle
\frac12\ln\left(\frac{\alpha_s^2+\xi_s^2}{\alpha_{s,0}^2+\xi_s^2}\right)\\
\\
b &=&\displaystyle \mbox{atan}\left(\frac{\alpha_s}{\xi_s}\right) -
\mbox{atan}\left(\frac{\alpha_{s,0}}{\xi_s}\right)\,,
\end{array}
\end{equation}
and thus:
\begin{equation}\label{AlphaEvolutionLO}
\begin{array}{l}
\displaystyle \exp\left[-R_0\Gamma_\alpha\right] = \\
\\
\displaystyle
\begin{pmatrix}
\displaystyle \mbox{cosh}\left(R_0\sqrt{a^2+b^2}\right)+\frac{a}{\sqrt{a^2+b^2}}\,\mbox{sinh}\left(R_0\sqrt{a^2+b^2}\right) & \displaystyle\frac{b}{\sqrt{a^2+b^2}}\,\mbox{sinh}\left(R_0\sqrt{a^2+b^2}\right) \\
\displaystyle\frac{b}{\sqrt{a^2+b^2}}\,\mbox{sinh}\left(R_0\sqrt{a^2+b^2}\right) & \displaystyle\mbox{cosh}\left(R_0\sqrt{a^2+b^2}\right)-\frac{a}{\sqrt{a^2+b^2}}\,\mbox{sinh}\left(R_0\sqrt{a^2+b^2}\right)
\end{pmatrix}\,.
\end{array}
\end{equation}

For $\Gamma_\xi$, instead, the structure is:
\begin{equation}
\Gamma_\alpha = \begin{pmatrix}
c & -d \\
d & c
\end{pmatrix}
\end{equation}
with:
\begin{equation}
\begin{array}{rcl}
c &=&\displaystyle
      \frac12\ln\left(\frac{\alpha_s^2+\xi_s^2}{\alpha_{s}^2+\xi_{s,0}^2}\right)\\
\\
d &=&\displaystyle \mbox{atan}\left(\frac{\xi_s}{\alpha_s}\right) -
\mbox{atan}\left(\frac{\xi_{s,0}}{\alpha_{s}}\right)\,,
\end{array}
\end{equation}
and thus, after some simplifications, we find:
\begin{equation}\label{XiEvolutionLO}
\exp\left[-R_0\Gamma_\xi\right] = \exp[-R_0c]
\begin{pmatrix}
\cos\left(R_0d\right) & -\sin\left(R_0d\right) \\
\sin\left(R_0d\right) & \cos\left(R_0d\right)
\end{pmatrix}\,.
\end{equation}
Eqs.~(\ref{AlphaEvolutionLO}) and~(\ref{XiEvolutionLO}) are the main
result of this section because they are the evolution factors in the
real and imaginary direction to be applied to the initial scale PDF.

At this point, in order to be able to implement in numerical code, it
is necessary to distinguish between non-slinglet and singlet
distributions. In the case of the non-singlet distributions
eqs.~(\ref{AlphaEvolutionLO}) and~(\ref{XiEvolutionLO}) can be
implemented exactly as they are written because the anomalous
dimension $\gamma^{(0)}$ appearing in
eqs.~(\ref{DGLAPcomplex1LOMellin}) and~(\ref{DGLAPcomplex2LOMellin})
is a singled-valued function. In the singlet case instead
$\gamma^{(0)}$ (and thus $R_0$) is actually a 2 by 2 matrix of functions and thus,
since it appears in the trigonometric functions,
eqs.~(\ref{AlphaEvolutionLO}) and~(\ref{XiEvolutionLO}) need to be
treated in a suitable way.

Given the definitions:
\begin{equation}
\begin{array}{rcl}
\mbox{sinh}(A) &=& \displaystyle\frac{\exp(A)-\exp(-A)}{2}\,,\\
\\
\mbox{cosh}(A) &=& \displaystyle\frac{\exp(A)+\exp(-A)}{2}\,,
\end{array}
\end{equation}
and:
\begin{equation}
\begin{array}{rcl}
\sin(A) &=& \displaystyle\frac{\exp(iA)-\exp(-iA)}{2i}\,,\\
\\ 
\cos(A) &=& \displaystyle\frac{\exp(iA)+\exp(-iA)}{2}\,,
\end{array}
\end{equation}
it is useful to rewrite eqs.~(\ref{AlphaEvolutionLO})
and~(\ref{XiEvolutionLO}) as follows:
\begin{equation}\label{AlphaEvolutionLOexp}
\begin{array}{l}
\displaystyle \exp\left[-R_0\Gamma_\alpha\right] = \\
\\
\displaystyle \frac{1}{2}\left[
\exp\left[R_0\sqrt{a^2+b^2}\right]\begin{pmatrix}
1 + \frac{a}{\sqrt{a^2+b^2}} & \frac{b}{\sqrt{a^2+b^2}} \\
\frac{b}{\sqrt{a^2+b^2}} & 1 + \frac{a}{\sqrt{a^2+b^2}}
\end{pmatrix}+
\exp\left[-R_0\sqrt{a^2+b^2}\right]
\begin{pmatrix}
1 - \frac{a}{\sqrt{a^2+b^2}} & -\frac{b}{\sqrt{a^2+b^2}} \\
-\frac{b}{\sqrt{a^2+b^2}} & 1 - \frac{a}{\sqrt{a^2+b^2}}
\end{pmatrix}
\right]\,,
\end{array}
\end{equation}
and:
\begin{equation}\label{XiEvolutionLOexp}
\exp\left[-R_0\Gamma_\xi\right] = \frac{\exp\left[-R_0c\right]}{2}\left[
\exp\left[iR_0d\right]\begin{pmatrix}
1 & i \\
-i & 1
\end{pmatrix}+
\exp\left[-iR_0d\right]\begin{pmatrix}
1 & -i \\
i & 1
\end{pmatrix}
\right]\,.
\end{equation}
This way one only needs to evaluate
$\exp\left[R_0\sqrt{a^2+b^2}\right]$ and $\exp\left[iR_0d\right]$ by
means of eqs.~(\ref{ExpMatrix})-(\ref{ExpMatrixEntries}) and take
their inverse (which an easy task) and evaluate
$\exp\left[-R_0c\right]$ by means of the same formulas. Finally, It
should be noticed that the matrices in the r.h.s. of
eqs.~(\ref{AlphaEvolutionLOexp}) and~(\ref{XiEvolutionLOexp}) are not
2 by 2 matrices but rather 4 by 4 because $R_0$ is a matrix in a
different space of the vector $\mathbf{F}$.

\end{document}
